\documentclass[aspectratio=141,10pt]{beamer}
%\usepackage{geometry}\geometry{a4paper,top=3.5cm,bottom=3.5cm,left=2.5cm,right=2.5cm,heightrounded,bindingoffset=0mm}
\usepackage[T1]{fontenc}
\usepackage[utf8]{inputenc}
\usepackage[italian]{babel}
\usepackage{graphicx}
\usepackage[export]{adjustbox}%Per il Frame attrono le immagini e il valign
\usepackage{subfig}
\usepackage{amsmath,amsfonts,amssymb,braket,mathrsfs}
\usepackage{float}
\usepackage{tabularx,booktabs}
\usepackage{hyperref}
\usepackage{epsfig}
\usepackage{pdfpages} %Per gli allegati
%\usepackage{minipage}
\usepackage[output-decimal-marker={,}]{siunitx}
\usepackage{tikz}
\usepackage{pgfplots,pgfplotstable}
%\pgfplotsset{compat=1.15} %indica la versione da utilizzare per pgfplot
\usetikzlibrary{patterns} % per il tratteggio
\usepgfplotslibrary{groupplots}
\pgfplotsset{compat=newest}
%\usepackage{stanli}
\usepackage{xspace}% per lo spazio intelligente
\newcommand{\e}{\`E\xspace}  %E'
\captionsetup{tableposition=top,figureposition=bottom,font=small}\captionsetup{format=hang,labelfont={bf,color=pantone186}} %didascalie a più righe allineate e il nome in grassetto
%non viene allineato a sinistra se la didascalia è corta una sola riga. PERCHé??
\usepackage{xcolor}
\usepackage{tcolorbox}
%serve per mettere il codice con lo sfondo grigio chiaro
\definecolor{pantone186}{RGB}{206, 17, 38} %il colore del logo UNITN
\definecolor{myGray}{gray}{0.5} %più basso più scuro è
\usepackage{listings} 
\lstset{basicstyle=\scriptsize\ttfamily,
backgroundcolor=\color{lightgray},%
boxpos=c,%
stringstyle=\itshape,		
lineskip=3pt,%
numbers=left,
numberstyle=\tiny,}
\usepackage{lscape}
\usepackage{multirow}
\usepackage{import}
%\usepackage{pythontex}
\usepackage[absolute,overlay]{textpos}
\usepackage{pgf} 
\title[Confronto prestazionale]{Confronto prestazionale tra diverse soluzioni edilizie}
\subtitle{d}
\author{Nicola Meoli}
\institute[UniTrento]{Università degli studi di Trento}
\date{22 dicembre 2020}
%\logo{\includegraphics{../IMG/logo_unitn_black_centerNEW.eps}}

\usefonttheme{serif} %font normali di latex
\usetheme[numbers,totalnumber]{madrid}
\usecolortheme{default}
\setbeamertemplate{navigation symbols}{} %rimuove la barra con le freccette
\setbeamercolor{title}{fg=white,bg=pantone186}
\setbeamercolor{titlelike}{fg=white,bg=pantone186}
\setbeamercolor{palette primary}{fg=white,bg=pantone186}
\setbeamercolor{palette secondary}{fg=white,bg=pantone186}
\setbeamercolor{palette tertiary}{fg=white,bg=pantone186}
\setbeamercolor{palette quaternary}{fg=white,bg=pantone186}
\setbeamercolor{palette sidebar primary}{fg=white,bg=pantone186}
\setbeamercolor{palette sidebar secondary}{fg=white,bg=pantone186}
\setbeamercolor{palette sidebar tertiary}{fg=white,bg=pantone186}
\setbeamercolor{palette sidebar quaternary}{fg=white,bg=pantone186}
\setbeamerfont{frametitle}{series=\bfseries} %titolo frame in grassetto
\begin{document}
\begin{frame}
\pagestyle{plain}
\thispagestyle{empty}
\begin{center}
  \begin{figure}[H]
    \centerline{\psfig{file=../IMG/logo_unitn_black_centerNEW.eps,
                        width=0.5\textwidth,trim = 0 1.5cm 0 0.6cm}}
  \end{figure}
\textcolor{pantone186}{\noindent\rule{0.7\textwidth}{.5pt}}

  \footnotesize{\textsc{Dipartimento di Ingegneria Civile, Ambientale e Meccanica\\}}
  \footnotesize{Corso di Laurea in Ingegneria Civile}

  \vspace{0.9 cm} 
  %\Large\textsc{Elaborato finale\\} 
  %\vspace{1 cm} 
  {\Large{\textsc{Confronto prestazionale\\ tra diverse soluzioni edilizie\\}}}
  
  \vspace{0.2 cm}
  {\it{Analisi termo igrometrica di alcuni pacchetti strutturali composti da differenti materiali. }}


  \vspace{1.5 cm} 
  \begin{tabular*}{\textwidth}{ l @{\extracolsep{\fill}} r }
  \textsc{Supervisore} & \textsc{Laureando}\\
  {Rossano Albatici}& {Nicola Meoli 186100}\\
  \end{tabular*}

  \vspace{.3cm} 
  \textcolor{pantone186}{\noindent\rule{\textwidth}{1pt}}
    
  {Anno accademico 2020/21}
\end{center}
\end{frame}

%---------------------------------------------------------------
%---------------------------------------------------------------

\begin{frame}{ciaoo}
    ciao 
    \pause
    ciaooo
\end{frame}
\begin{frame}
    \frametitle{Verifica condensa interstiziale}
    \framesubtitle{Temperatura}
Calcolo della temperatura per ogni strato a partire dalla temperatura interna 
    \begin{align*}
        \Delta\vartheta &= \vartheta_{\text{int.}} - \vartheta_{\text{est.}} \\
        \vartheta^{(1)} &= \vartheta_{\text{int.}} \\
        \vartheta^{(k)} &= \vartheta_{\text{int.}} - \frac{\Delta\vartheta}{R_{tot}} \, \left(R_{\text{s,int.}} + \sum_{i=1}^{k}R^{(i)}\right) \\
        \vartheta^{(N)} &= \vartheta_{\text{est.}} \qquad \left[\SI{}{\celsius}\right]
    \end{align*}
    \pause
    ciaooo
\end{frame}
\begin{frame}
    \frametitle{Confronto verifica condensa interstizionale}
    \framesubtitle{Parete solo mattone pieno (0a)}
    \begin{columns}
        \begin{column}{0.45\textwidth}
            \resizebox{\textwidth}{!}{%
            \input{tabella-input-0a.tex}
            }
        \end{column}
        \pause
        \begin{column}{0.55\textwidth}
            \scriptsize
            \input{temperatura+pressione--parete-0a.tex}
        \end{column}
    \end{columns}
\end{frame}
\begin{frame}
    \frametitle{Confronto verifica condensa interstizionale}
    \framesubtitle{Parete solo isolante (0b)}
    \begin{columns}
        \begin{column}{0.45\textwidth}
            \resizebox{\textwidth}{!}{%
            \begin{table}[H]
\centering
\begin{tabular}{lS[table-format=1.3]S[table-format=1.3]S[table-format=2.1]}
\toprule
\multicolumn{1}{c}{\multirow{3}{*}{Strati}} & \multicolumn{1}{c}{Spessori} & \multicolumn{1}{c}{Conducibilità} & \multicolumn{1}{c}{Permeabilità} \\
\multicolumn{1}{c}{} & \multicolumn{1}{c}{$s$} & \multicolumn{1}{c}{termica $\lambda$} & \multicolumn{1}{c}{al vapore $\mu$} \\
\multicolumn{1}{c}{} & \multicolumn{1}{c}{[\SI{}{\metre}]} & \multicolumn{1}{c}{[\SI{}{\watt\per\metre\squared\per\kelvin}]} & \multicolumn{1}{c}{$[-]$} \\
\midrule
 Isolante VentirockDuo  &     0,12 &         0,035 &  1,0 \\
\bottomrule
\end{tabular}
\end{table}

            }
        \end{column}
        \begin{column}{0.55\textwidth}
            \scriptsize
            \input{temperatura+pressione--parete-0b.tex}
        \end{column}
    \end{columns}
\end{frame}
\begin{frame}
    \frametitle{Confronto verifica condensa interstizionale}
    \framesubtitle{Parete in laterizio con isolante interno (1a)}
    \begin{columns}
        \begin{column}{0.45\textwidth}
            \resizebox{\textwidth}{!}{%
            \begin{tabular}{lS[table-format=1.3]S[table-format=1.3]S[table-format=2.1]}
\toprule
\multicolumn{1}{c}{\multirow{3}{*}{Strati}} & \multicolumn{1}{c}{Spessori} & \multicolumn{1}{c}{Conducibilità} & \multicolumn{1}{c}{Permeabilità} \\
\multicolumn{1}{c}{} & \multicolumn{1}{c}{$s$} & \multicolumn{1}{c}{termica $\lambda$} & \multicolumn{1}{c}{al vapore $\mu$} \\
\multicolumn{1}{c}{} & \multicolumn{1}{c}{[\SI{}{\metre}]} & \multicolumn{1}{c}{[\SI{}{\watt\per\metre\squared\per\kelvin}]} & \multicolumn{1}{c}{$[-]$} \\
\midrule
              Intonaco &    0,015 &         0,600 &   6,0 \\
 Isolante VentirockDuo &    0,120 &         0,035 &   1,0 \\
   Laterizio semipieno &    0,200 &         0,530 &  10,0 \\
              Intonaco &    0,015 &         0,900 &   8,0 \\
\bottomrule
\end{tabular}

            }
        \end{column}
        \begin{column}{0.55\textwidth}
            \scriptsize
            % This file was created by tikzplotlib v0.9.4.
\begin{tikzpicture}

\definecolor{color0}{rgb}{1,0.647058823529412,0}

\begin{groupplot}[group style={group size=1 by 2,vertical sep=2.5cm}]
\nextgroupplot[
	 ticklabel style={ 
 		 /pgf/number format/fixed, 
 		 /pgf/number format/precision=5
		}, 
scaled ticks=false,
height=7cm,
legend cell align={left},
legend style={fill opacity=0.8, draw opacity=1, text opacity=1, draw=white!80!black},
minor xtick={},
minor ytick={},
tick align=outside,
tick pos=left,
width=\linewidth,
x grid style={white!69.0196078431373!black},
xlabel={Spessore parete (m)},
xmajorgrids,
xmin=-0.0185, xmax=0.3885,
xtick style={color=black},
xtick={0,0.015,0.155,0.355,0.37},
xticklabel style = {rotate=90.0},
y grid style={white!69.0196078431373!black},
ylabel={Temperature (°C)},
ymin=-6.01137676302392, ymax=20.5685707562642,
ytick style={color=black},
ytick={-10,-5,0,5,10,15,20,25}
]
\addplot [line width=1.5pt, green!50.1960784313725!black]
table {%
0 19.3603913235693
0.015 19.2373896550249
0.155 -0.442877312074838
0.355 -4.72119621796609
0.37 -4.803197330329
};
\addlegendentry{Temperatura}

\nextgroupplot[
	 ticklabel style={ 
 		 /pgf/number format/fixed, 
 		 /pgf/number format/precision=5
		}, 
scaled ticks=false,
height=7cm,
legend cell align={left},
legend style={fill opacity=0.8, draw opacity=1, text opacity=1, draw=white!80!black},
minor xtick={},
minor ytick={},
tick align=outside,
tick pos=left,
width=\linewidth,
x grid style={white!69.0196078431373!black},
xlabel={Spessore equivalente Sd (m)},
xmajorgrids,
xmin=-0.1175, xmax=2.4675,
xtick style={color=black},
xtick={0,0.09,0.23,2.23,2.35},
xticklabel style = {rotate=90.0},
y grid style={white!69.0196078431373!black},
ylabel={Pressione (Pa)},
ymin=266.815315862144, ymax=2340.26179765774,
ytick style={color=black},
ytick={250,500,750,1000,1250,1500,1750,2000,2250,2500}
]
\addplot [line width=1.5pt, color0]
table {%
0 1519.01824347152
0.09 1474.67101690856
0.23 1405.68644225507
2.23 420.192518633768
2.35 361.06288321649
};
\addlegendentry{Pressione}
\addplot [line width=1.5pt, red]
table {%
0 2246.0142303034
0.09 2228.8857975183
0.23 588.588856594798
2.23 410.858360238758
2.35 407.990247749789
};
\addlegendentry{Pressione Saturazione}
\end{groupplot}

\end{tikzpicture}

        \end{column}
    \end{columns}
\end{frame}
\begin{frame}
    \frametitle{Confronto verifica condensa interstizionale}
    \framesubtitle{Parete in laterizio con isolante esterno (1b)}
    \begin{columns}
        \begin{column}{0.45\textwidth}
            \resizebox{\textwidth}{!}{%
            \begin{table}[H]
\centering
\caption{Parete in muratura con isolante esterno}
\begin{tabular}{lrrr}
\toprule
            Strati & Spessori & Conduttività  &    mu \\
\midrule
          Intonaco &    0,015 &                0,600 &   6,0 \\
 Laterizio Poroton &    0,200 &                0,230 &  10,0 \\
  Isolante RockDuo &    0,140 &                0,035 &   1,0 \\
          Intonaco &    0,015 &                0,900 &   8,0 \\
\bottomrule
\end{tabular}
\end{table}

            }
        \end{column}
        \begin{column}{0.55\textwidth}
            \scriptsize
            \input{temperatura+pressione--parete-1b.tex}
        \end{column}
    \end{columns}
\end{frame}
\begin{frame}
    \frametitle{Confronto verifica condensa interstizionale}
    \framesubtitle{Parete in muratura Poroton con isolante esterno (2a)}
    \begin{columns}
        \begin{column}{0.45\textwidth}
            \resizebox{\textwidth}{!}{%
            \input{tabella-input-2a.tex}
            }
        \end{column}
        \begin{column}{0.55\textwidth}
            \scriptsize
            \input{temperatura+pressione--parete-2a.tex}
        \end{column}
    \end{columns}
\end{frame}
\begin{frame}
    \frametitle{Confronto verifica condensa interstizionale}
    \framesubtitle{Parete in X-LAM con isolante bassa densità (3a)}
    \begin{columns}
        \begin{column}{0.45\textwidth}
            \resizebox{\textwidth}{!}{%
            \begin{table}[H]
\centering
\caption{Perete in X-LAM con isolante bassa densità}
\begin{tabular}{lrrr}
\toprule
                      Strati & Spessori & Conduttività  &    mu \\
\midrule
                  Gessofibra &    0,013 &                0,210 &   5,0 \\
                   X-LAM KLH &    0,096 &                0,130 &  25,0 \\
 Isolante bassa densità  &    0,120 &                0,038 &   1,0 \\
              Intonaco calce &    0,015 &                0,900 &  20,0 \\
\bottomrule
\end{tabular}
\end{table}

            }
        \end{column}
        \begin{column}{0.55\textwidth}
            \scriptsize
            % This file was created by tikzplotlib v0.9.4.
\begin{tikzpicture}

\definecolor{color0}{rgb}{1,0.647058823529412,0}

\begin{groupplot}[group style={group size=1 by 2,vertical sep=2.5cm}]
\nextgroupplot[
	 ticklabel style={ 
 		 /pgf/number format/fixed, 
 		 /pgf/number format/precision=5
		}, 
scaled ticks=false,
height=7cm,
legend cell align={left},
legend style={fill opacity=0.8, draw opacity=1, text opacity=1, draw=white!80!black},
minor xtick={},
minor ytick={},
tick align=outside,
tick pos=left,
width=\linewidth,
x grid style={white!69.0196078431373!black},
xlabel={Spessore parete (m)},
xmajorgrids,
xmin=-0.0122, xmax=0.2562,
xtick style={color=black},
xtick={0,0.013,0.109,0.229,0.244},
xticklabel style = {rotate=90.0},
y grid style={white!69.0196078431373!black},
ylabel={Temperature (°C)},
ymin=-5.95747378805511, ymax=20.414641642178,
ytick style={color=black},
ytick={-10,-5,0,5,10,15,20,25}
]
\addplot [line width=1.5pt, green!50.1960784313725!black]
table {%
0 19.2159091226219
0.013 18.8425325143467
0.109 14.3885251635601
0.229 -4.65821679704034
0.244 -4.75874126849906
};
\addlegendentry{Temperatura}

\nextgroupplot[
	 ticklabel style={ 
 		 /pgf/number format/fixed, 
 		 /pgf/number format/precision=5
		}, 
scaled ticks=false,
height=7cm,
legend cell align={left},
legend style={fill opacity=0.8, draw opacity=1, text opacity=1, draw=white!80!black},
minor xtick={},
minor ytick={},
tick align=outside,
tick pos=left,
width=\linewidth,
x grid style={white!69.0196078431373!black},
xlabel={Spessore equivalente Sd (m)},
xmajorgrids,
xmin=-0.14425, xmax=3.02925,
xtick style={color=black},
xtick={0,0.065,2.465,2.585,2.885},
xticklabel style = {rotate=90.0},
y grid style={white!69.0196078431373!black},
ylabel={Pressione (Pa)},
ymin=267.820712152638, ymax=2319.14847555738,
ytick style={color=black},
ytick={250,500,750,1000,1250,1500,1750,2000,2250,2500}
]
\addplot [line width=1.5pt, color0]
table {%
0 1519.01824347152
0.065 1492.92912790252
2.465 529.638706893132
2.585 481.474185842663
2.885 361.06288321649
};
\addlegendentry{Pressione}
\addplot [line width=1.5pt, red]
table {%
0 2225.90630449353
0.065 2174.67021383189
2.465 1638.46020552403
2.585 413.073607008036
2.885 409.542895587558
};
\addlegendentry{Pressione Saturazione}
\end{groupplot}

\end{tikzpicture}

        \end{column}
    \end{columns}
\end{frame}
\begin{frame}
    \frametitle{Confronto verifica condensa interstizionale}
    \framesubtitle{Parete in X-LAM con isolante bassa densità lana di roccia (3b)}
    \begin{columns}
        \begin{column}{0.45\textwidth}
            \resizebox{\textwidth}{!}{%
            \begin{table}[H]
\centering
\caption{Perete in X-LAM con isolante bassa densità lana di roccia}
\begin{tabular}{lrrr}
\toprule
                Strati & Spessori & Conduttività &    mu \\
\midrule
            Gessofibra &    0,013 &                0,210 &   5,0 \\
             X-LAM KLH &    0,096 &                0,130 &  25,0 \\
 Isolante ventirockduo &    0,105 &                0,035 &   1,0 \\
        Intonaco calce &    0,015 &                0,900 &  20,0 \\
\bottomrule
\end{tabular}
\end{table}

            }
        \end{column}
        \begin{column}{0.55\textwidth}
            \scriptsize
            % This file was created by tikzplotlib v0.9.4.
\begin{tikzpicture}

\definecolor{color0}{rgb}{1,0.647058823529412,0}

\begin{groupplot}[group style={group size=1 by 2,vertical sep=2.5cm}]
\nextgroupplot[
	 ticklabel style={ 
 		 /pgf/number format/fixed, 
 		 /pgf/number format/precision=5
		}, 
scaled ticks=false,
height=7cm,
legend cell align={left},
legend style={fill opacity=0.8, draw opacity=1, text opacity=1, draw=white!80!black},
minor xtick={},
minor ytick={},
tick align=outside,
tick pos=left,
width=\linewidth,
x grid style={white!69.0196078431373!black},
xlabel={Spessore parete (m)},
xmajorgrids,
xmin=-0.01145, xmax=0.24045,
xtick style={color=black},
xtick={0,0.013,0.109,0.214,0.229},
xticklabel style = {rotate=90.0},
y grid style={white!69.0196078431373!black},
ylabel={Temperature (°C)},
ymin=-5.94588914613306, ymax=20.3815597265862,
ytick style={color=black},
ytick={-10,-5,0,5,10,15,20,25}
]
\addplot [line width=1.5pt, green!50.1960784313725!black]
table {%
0 19.1848575050989
0.013 18.7966944122889
0.109 14.1662991749812
0.214 -4.64468147658159
0.229 -4.74918692464583
};
\addlegendentry{Temperatura}

\nextgroupplot[
	 ticklabel style={ 
 		 /pgf/number format/fixed, 
 		 /pgf/number format/precision=5
		}, 
scaled ticks=false,
height=7cm,
legend cell align={left},
legend style={fill opacity=0.8, draw opacity=1, text opacity=1, draw=white!80!black},
minor xtick={},
minor ytick={},
tick align=outside,
tick pos=left,
width=\linewidth,
x grid style={white!69.0196078431373!black},
xlabel={Spessore equivalente Sd (m)},
xmajorgrids,
xmin=-0.1435, xmax=3.0135,
xtick style={color=black},
xtick={0,0.065,2.465,2.57,2.87},
xticklabel style = {rotate=90.0},
y grid style={white!69.0196078431373!black},
ylabel={Pressione (Pa)},
ymin=268.035757521022, ymax=2314.63252282132,
ytick style={color=black},
ytick={250,500,750,1000,1250,1500,1750,2000,2250,2500}
]
\addplot [line width=1.5pt, color0]
table {%
0 1519.01824347152
0.065 1492.79277363996
2.465 524.46773370544
2.57 482.103513208305
2.87 361.06288321649
};
\addlegendentry{Pressione}
\addplot [line width=1.5pt, red]
table {%
0 2221.60539712585
0.065 2168.45182859209
2.465 1615.05395862344
2.57 413.551117171863
2.87 409.87728665048
};
\addlegendentry{Pressione Saturazione}
\end{groupplot}

\end{tikzpicture}

        \end{column}
    \end{columns}
\end{frame}
\begin{frame}
    \frametitle{Confronto verifica condensa interstizionale}
    \framesubtitle{Parete in X-LAM con isolante alta densità fibra di legno (3c)}
    \begin{columns}
        \begin{column}{0.45\textwidth}
            \resizebox{\textwidth}{!}{%
            \begin{table}[H]
\centering
\caption{Perete in X-LAM con isolante alta densità fibra di legno}
\begin{tabular}{lrrr}
\toprule
                                    Strati & Spessori & Conduttività  &    mu \\
\midrule
                                Gessofibra &    0,013 &                0,210 &   5,0 \\
                                 X-LAM KLH &    0,096 &                0,130 &  25,0 \\
 Isolante alta densità &    0,130 &                0,043 &   5,0 \\
                            Intonaco calce &    0,015 &                0,900 &  20,0 \\
\bottomrule
\end{tabular}
\end{table}

            }
        \end{column}
        \begin{column}{0.55\textwidth}
            \scriptsize
            % This file was created by tikzplotlib v0.9.4.
\begin{tikzpicture}

\definecolor{color0}{rgb}{1,0.647058823529412,0}

\begin{groupplot}[group style={group size=1 by 2,vertical sep=2.5cm}]
\nextgroupplot[
	 ticklabel style={ 
 		 /pgf/number format/fixed, 
 		 /pgf/number format/precision=5
		}, 
scaled ticks=false,
height=7cm,
legend cell align={left},
legend style={fill opacity=0.8, draw opacity=1, text opacity=1, draw=white!80!black},
minor xtick={},
minor ytick={},
tick align=outside,
tick pos=left,
width=\linewidth,
x grid style={white!69.0196078431373!black},
xlabel={Spessore parete (m)},
xmajorgrids,
xmin=-0.0127, xmax=0.2667,
xtick style={color=black},
xtick={0,0.013,0.109,0.239,0.254},
xticklabel style = {rotate=90.0},
y grid style={white!69.0196078431373!black},
ylabel={Temperature (°C)},
ymin=-5.94765269629741, ymax=20.3865958440658,
ytick style={color=black},
ytick={-10,-5,0,5,10,15,20,25}
]
\addplot [line width=1.5pt, green!50.1960784313725!black]
table {%
0 19.1895845467766
0.013 18.803672426194
0.109 14.2001290232679
0.239 -4.64674198192825
0.254 -4.75064139900817
};
\addlegendentry{Temperatura}

\nextgroupplot[
	 ticklabel style={ 
 		 /pgf/number format/fixed, 
 		 /pgf/number format/precision=5
		}, 
scaled ticks=false,
height=7cm,
legend cell align={left},
legend style={fill opacity=0.8, draw opacity=1, text opacity=1, draw=white!80!black},
minor xtick={},
minor ytick={},
tick align=outside,
tick pos=left,
width=\linewidth,
x grid style={white!69.0196078431373!black},
xlabel={Spessore equivalente Sd (m)},
xmajorgrids,
xmin=-0.17075, xmax=3.58575,
xtick style={color=black},
xtick={0,0.065,2.465,3.115,3.415},
xticklabel style = {rotate=90.0},
y grid style={white!69.0196078431373!black},
ylabel={Pressione (Pa)},
ymin=268.003044265693, ymax=2315.31950118321,
ytick style={color=black},
ytick={250,500,750,1000,1250,1500,1750,2000,2250,2500}
]
\addplot [line width=1.5pt, color0]
table {%
0 1519.01824347152
0.065 1496.97809752231
2.465 683.188093243512
3.115 462.786633751339
3.415 361.06288321649
};
\addlegendentry{Pressione}
\addplot [line width=1.5pt, red]
table {%
0 2222.25966223242
0.065 2169.39745901465
2.465 1618.59809228845
3.115 413.4783925652
3.415 409.826365687118
};
\addlegendentry{Pressione Saturazione}
\end{groupplot}

\end{tikzpicture}

        \end{column}
    \end{columns}
\end{frame}



% ---------------------
% dinamica
% ----------------------
\begin{frame}
    \frametitle{Confronto}
    \framesubtitle{Parete solo mattone pieno (0a)}
    \input{../CHAPTERS/tabella-dinamica-0a.tex}
\end{frame}
\begin{frame}
    \frametitle{Confronto}
    \framesubtitle{Parete solo isolante (0b)}
    \input{../CHAPTERS/tabella-dinamica-0b.tex}
\end{frame}
\begin{frame}
    \frametitle{Confronto}
    \framesubtitle{Parete in laterizio con isolante interno (1a)}
    \begin{table}[H]
\centering
\resizebox{\linewidth}{!}{%
\begin{tabular}{@{}
				l
				S[table-format=1.3]
				S[table-format=4.1]
				S[table-format=4.1]
				S[table-format=2.2]
				S[table-format=1.3]
				S[table-format=1.3]
				S[table-format=2.1]
				S[table-format=1.1]
				S[table-format=4.1]
				@{}
				}
\toprule
\multicolumn{1}{c}{\multirow{3}{*}{Strati}} & \multicolumn{1}{c}{\multirow{2}{*}{Spessori}}    & \multicolumn{1}{c}{\multirow{2}{*}{Densità}}                                            & \multicolumn{1}{c}{Calore}                                            & \multicolumn{1}{c}{Massa}                                               & \multicolumn{1}{c}{Profondità di}                      & \multicolumn{1}{c}{Rapporto}   & \multicolumn{1}{c}{Capacità}                                                  & \multicolumn{1}{c}{Diffusività}                                      & \multicolumn{1}{c}{Effusività}                                                             \\
\multicolumn{1}{c}{}                        & \multicolumn{1}{c}{}                              & \multicolumn{1}{c}{}                                                  & \multicolumn{1}{c}{specifico}                                         & \multicolumn{1}{c}{superficiale}                                        & \multicolumn{1}{c}{penetrazione $\delta$} & \multicolumn{1}{c}{$\xi$}      & \multicolumn{1}{c}{termica areica}                                            & \multicolumn{1}{c}{termica}                                          & \multicolumn{1}{c}{Termica}                                                                \\
\multicolumn{1}{c}{}                        & \multicolumn{1}{c}{$\left[\SI{}{\metre}\right]$} & \multicolumn{1}{c}{$\left[\SI{}{\kilo\gram\per\metre\cubed}\right]$} & \multicolumn{1}{c}{$\left[\SI{}{\joule\per\kilo\gram\per\kelvin}\right]$} & \multicolumn{1}{c}{$\left[\SI{}{\kilo\gram\per\metre\squared}\right]$} & \multicolumn{1}{c}{$\left[\SI{}{\metre}\right]$}      & \multicolumn{1}{c}{$[-]$} & \multicolumn{1}{c}{$\left[\SI{}{\kilo\joule\per\metre\squared\per\kelvin}\right]$} & \multicolumn{1}{c}{$\left[\SI{.e-7}{\metre\squared\per\second}\right]$} & \multicolumn{1}{c}{$\left[\SI{}{\watt\second\tothe{0.5}\per\metre\squared\per\kelvin}\right]$} \\
\midrule
   Intonaco              & 0,015 & 1500,0 & 1000,0 & 22,5  & 0,105 & 0,143 & 22,5  & 4,00 & 948,7 \\
   Isolante VentirockDuo & 0,120 & 70,0   & 1030,0 & 8,4   & 0,116 & 1,039 & 8,7   & 4,85 & 50,2 \\
   Laterizio semipieno   & 0,200 & 1000,0 & 840,0  & 200,0 & 0,132 & 1,518 & 168,0 & 6,31 & 667,2 \\
   Intonaco              & 0,015 & 1800,0 & 1000,0 & 27,0  & 0,117 & 0,128 & 27,0  & 5,00 & 1272,8 \\
\bottomrule
\end{tabular}%
}
\end{table}

\begin{flushleft}
\begin{align*}
\text{Massa superficiale totale} \, M_s &= \SI{257.9}{\kilo\gram\per\metre\squared}\\
\text{Sfasamento} \, \Delta\tau &= \SI{9.09}{\hour}\\
\text{Fattore di attenuazione} \, fd &= \SI{0.352}{}\\
\text{Trasmittanza termica periodica} \, Y_{12} &= \SI{0.088}{\watt\per\metre\squared\per\kelvin}\\
\text{Ammettanza termica interna} \, Y_{11} &= \SI{1.744}{\watt\per\metre\squared\per\kelvin}\\
\text{Ammettanza termica esterna} \, Y_{22} &= \SI{5.991}{\watt\per\metre\squared\per\kelvin}\\
\text{Capacità termica periodica interna} \, k_1 &= \SI{25.09}{\kilo\joule\per\metre\squared\per\kelvin}\\
\end{align*}
\end{flushleft}

\end{frame}
\begin{frame}
    \frametitle{Confronto}
    \framesubtitle{Parete in laterizio con isolante esterno (1b)}
    \begin{table}
\centering
\caption{Parete in muratura con isolante esterno}
\begin{tabular}{lrrrrrr}
\toprule
            Strati & Spessori & Densità & Calore specifico & Massa superficiale & Profondità di Penetrazione &     xi \\
\midrule
          Intonaco &    0,015 &  1500,0 &           1000,0 &               22,5 &                      0,105 &  0,143 \\
 Laterizio Poroton &    0,200 &   860,0 &            840,0 &              172,0 &                      0,094 &  2,137 \\
  Isolante RockDuo &    0,140 &  1200,0 &           1500,0 &              168,0 &                      0,023 &  6,054 \\
          Intonaco &    0,015 &  1800,0 &           1000,0 &               27,0 &                      0,117 &  0,128 \\
\bottomrule
\end{tabular}
\end{table}

\begin{flushleft}
\begin{align*}
\text{Massa superficiale totale} &= \SI{389.5}{\kilo\gram}\\
\text{Sfasamento} &= \SI{6.55}{\hour}\\
\text{Attenuazione} &= \SI{304.872}{}
\end{align*}
\end{flushleft}

\end{frame}
\begin{frame}
    \frametitle{Confronto}
    \framesubtitle{Parete in muratura Poroton con isolante esterno (2a)}
    \begin{table}[H]
\centering
\resizebox{\linewidth}{!}{%
\begin{tabular}{@{}
				l
				S[table-format=1.3]
				S[table-format=4.1]
				S[table-format=4.1]
				S[table-format=2.2]
				S[table-format=1.3]
				S[table-format=1.3]
				S[table-format=2.1]
				S[table-format=1.1]
				S[table-format=4.1]
				@{}
				}
\toprule
\multicolumn{1}{c}{\multirow{3}{*}{Strati}} & \multicolumn{1}{c}{\multirow{2}{*}{Spessori}}    & \multicolumn{1}{c}{\multirow{2}{*}{Densità}}                                            & \multicolumn{1}{c}{Calore}                                            & \multicolumn{1}{c}{Massa}                                               & \multicolumn{1}{c}{Profondità di}                      & \multicolumn{1}{c}{Rapporto}   & \multicolumn{1}{c}{Capacità}                                                  & \multicolumn{1}{c}{Diffusività}                                      & \multicolumn{1}{c}{Effusività}                                                             \\
\multicolumn{1}{c}{}                        & \multicolumn{1}{c}{}                              & \multicolumn{1}{c}{}                                                  & \multicolumn{1}{c}{specifico}                                         & \multicolumn{1}{c}{superficiale}                                        & \multicolumn{1}{c}{penetrazione $\delta$} & \multicolumn{1}{c}{$\xi$}      & \multicolumn{1}{c}{termica areica}                                            & \multicolumn{1}{c}{termica}                                          & \multicolumn{1}{c}{Termica}                                                                \\
\multicolumn{1}{c}{}                        & \multicolumn{1}{c}{$\left[\SI{}{\metre}\right]$} & \multicolumn{1}{c}{$\left[\SI{}{\kilo\gram\per\metre\cubed}\right]$} & \multicolumn{1}{c}{$\left[\SI{}{\joule\per\kilo\gram\per\kelvin}\right]$} & \multicolumn{1}{c}{$\left[\SI{}{\kilo\gram\per\metre\squared}\right]$} & \multicolumn{1}{c}{$\left[\SI{}{\metre}\right]$}      & \multicolumn{1}{c}{$[-]$} & \multicolumn{1}{c}{$\left[\SI{}{\kilo\joule\per\metre\squared\per\kelvin}\right]$} & \multicolumn{1}{c}{$\left[\SI{.e-7}{\metre\squared\per\second}\right]$} & \multicolumn{1}{c}{$\left[\SI{}{\watt\second\tothe{0.5}\per\metre\squared\per\kelvin}\right]$} \\
\midrule
              Intonaco &    0,015 &  1500,0 &           1000,0 &               22,5 &                      0,105 &  0,143 &             22,5 &                       4,00 &              948,7 \\
     Laterizio Poroton &    0,200 &   860,0 &            840,0 &              172,0 &                      0,094 &  2,137 &            144,5 &                       3,18 &              407,6 \\
 Isolante VentirockDuo &    0,110 &    70,0 &           1030,0 &                7,7 &                      0,116 &  0,952 &              7,9 &                       4,85 &               50,2 \\
              Intonaco &    0,015 &  1800,0 &           1000,0 &               27,0 &                      0,117 &  0,128 &             27,0 &                       5,00 &             1272,8 \\
\bottomrule
\end{tabular}%
}
\end{table}

\begin{flushleft}
\begin{align*}
\text{Massa superficiale totale} \, M_s &= \SI{229.2}{\kilo\gram\per\metre\squared}\\
\text{Sfasamento} \, \Delta\tau &= \SI{11.25}{\hour}\\
\text{Fattore di attenuazione} \, fd &= \SI{0.182}{}\\
\text{Trasmittanza termica periodica} \, Y_{12} &= \SI{0.043}{\watt\per\metre\squared\per\kelvin}\\
\text{Ammettanza termica interna} \, Y_{11} &= \SI{3.171}{\watt\per\metre\squared\per\kelvin}\\
\text{Ammettanza termica esterna} \, Y_{22} &= \SI{2.154}{\watt\per\metre\squared\per\kelvin}\\
\text{Capacità termica periodica interna} \, k_1 &= \SI{44.15}{\kilo\joule\per\metre\squared\per\kelvin}\\
\end{align*}
\end{flushleft}

\end{frame}
\begin{frame}
    \frametitle{Confronto}
    \framesubtitle{Parete in X-LAM con isolante bassa densità (3a)}
    \begin{table}
\centering
\caption{Perete in X-LAM con isolante bassa densità NOME}
\begin{tabular}{lrrrrrr}
\toprule
                      Strati & Spessori & Densità & Calore specifico & Massa superficiale & Profondità di Penetrazione &     xi \\
\midrule
                  Gessofibra &    0,013 &  1150,0 &           1100,0 &              14,95 &                      0,068 &  0,192 \\
                   X-LAM KLH &    0,096 &   500,0 &           1600,0 &              48,00 &                      0,067 &  1,436 \\
 Isolante bassa densità NOME &    0,120 &    50,0 &           2100,0 &               6,00 &                      0,100 &  1,203 \\
              Intonaco calce &    0,015 &  1800,0 &           1000,0 &              27,00 &                      0,117 &  0,128 \\
\bottomrule
\end{tabular}
\end{table}

\begin{flushleft}
\begin{align*}
\text{Massa superficiale totale} &= \SI{95.95}{\kilo\gram}\\
\text{Sfasamento} &= \SI{9.21}{\hour}\\
\text{Attenuazione} &= \SI{248.694}{}
\end{align*}
\end{flushleft}

\end{frame}
\begin{frame}
    \frametitle{Confronto}
    \framesubtitle{Parete in X-LAM con isolante bassa densità lana di roccia (3b)}
    \begin{table}
\centering
\caption{Perete in X-LAM con isolante bassa densità lana di roccia}
\begin{tabular}{lrrrrrr}
\toprule
                Strati & Spessori & Densità & Calore specifico & Massa superficiale & Profondità di Penetrazione &     xi \\
\midrule
            Gessofibra &    0,013 &  1150,0 &           1100,0 &              14,95 &                      0,068 &  0,192 \\
             X-LAM KLH &    0,096 &   500,0 &           1600,0 &              48,00 &                      0,067 &  1,436 \\
 Isolante ventirockduo &    0,105 &    70,0 &           1030,0 &               7,35 &                      0,116 &  0,909 \\
        Intonaco calce &    0,015 &  1800,0 &           1000,0 &              27,00 &                      0,117 &  0,128 \\
\bottomrule
\end{tabular}
\end{table}

\begin{flushleft}
\begin{align*}
\text{Massa superficiale totale} &= \SI{97.3}{\kilo\gram}\\
\text{Sfasamento} &= \SI{8.34}{\hour}\\
\text{Attenuazione} &= \SI{239.221}{}
\end{align*}
\end{flushleft}

\end{frame}
\begin{frame}
    \frametitle{Confronto}
    \framesubtitle{Parete in X-LAM con isolante alta densità fibra di legno (3c)}
    \begin{table}
\centering
\caption{Perete in X-LAM con isolante alta densità fibra di legno}
\begin{tabular}{lrrrrrr}
\toprule
                                    Strati & Spessori & Densità & Calore specifico & Massa superficiale & Profondità di Penetrazione &     xi \\
\midrule
                                Gessofibra &    0,013 &  1150,0 &           1100,0 &              14,95 &                      0,068 &  0,192 \\
                                 X-LAM KLH &    0,096 &   500,0 &           1600,0 &              48,00 &                      0,067 &  1,436 \\
 Isolante alta densità Naturalia Diffuterm &    0,130 &   190,0 &           2100,0 &              24,70 &                      0,054 &  2,388 \\
                            Intonaco calce &    0,015 &  1800,0 &           1000,0 &              27,00 &                      0,117 &  0,128 \\
\bottomrule
\end{tabular}
\end{table}

\begin{flushleft}
\begin{align*}
\text{Massa superficiale totale} &= \SI{114.65}{\kilo\gram}\\
\text{Sfasamento} &= \SI{13.65}{\hour}\\
\text{Attenuazione} &= \SI{240.616}{}
\end{align*}
\end{flushleft}

\end{frame}
% -------------------
% grafici confronti
% -------------------
\begin{frame}
    \frametitle{Confronto totale}
    \framesubtitle{...che fuoriesce dallo schermo}
    \centering
    \resizebox{0.9\linewidth}{!}{%
    \footnotesize
        \input{confrontobeamerVersion.tex}
    }
    \begin{textblock*}{15cm}(4.8cm,0.1cm)
        \scriptsize
            \definecolor{color0}{rgb}{0.12156862745098,0.466666666666667,0.705882352941177}
            \definecolor{color1}{rgb}{1,0.498039215686275,0.0549019607843137}
            \definecolor{color2}{rgb}{0.172549019607843,0.627450980392157,0.172549019607843}
            \definecolor{color3}{rgb}{0.83921568627451,0.152941176470588,0.156862745098039}
            \definecolor{color4}{rgb}{0.580392156862745,0.403921568627451,0.741176470588235}
            \definecolor{color5}{rgb}{0.549019607843137,0.337254901960784,0.294117647058824}
            \definecolor{color6}{rgb}{0.890196078431372,0.466666666666667,0.76078431372549}
            \begin{flushleft}
                {%
                \setlength{\fboxrule}{1pt}
                \fcolorbox{color0}{white}{Spessore $[\SI{}{\metre}]$} 
                \fcolorbox{color1}{white}{Resistenza $[\SI{}{\metre\squared\kelvin\per\watt}]$} 
                \fcolorbox{color2}{white}{Massa superficiale $[\times\SI{.e2}{\kilo\gram\per\metre\squared}]$} \\
                \fcolorbox{color3}{white}{Trasmittanza termica periodica $Y_{12} [\SI{}{\watt\per\metre\squared\per\kelvin}]$} 
                \fcolorbox{color4}{white}{Sfasamento $[\SI{}{\hour}]$} \\
                \fcolorbox{color5}{white}{Fattore di attenuazione $[\times\SI{.e-1}{}]$} \\
                \fcolorbox{color6}{white}{Capacità termica periodica interna $k_1 [\times \SI{.e-1}{\kilo\joule\per\metre\squared\per\kelvin}]$} 
                }
            \end{flushleft}
        \end{textblock*}
\end{frame}
\begin{frame}
    \frametitle{Confronto totale}
    \framesubtitle{...senza le prime due pareti}
    \centering
    \resizebox{0.9\linewidth}{!}{%
    \footnotesize
        \input{confronto2beamerVersion.tex}
    }
    \begin{textblock*}{15cm}(4.8cm,0.1cm)
        \scriptsize
            \definecolor{color0}{rgb}{0.12156862745098,0.466666666666667,0.705882352941177}
            \definecolor{color1}{rgb}{1,0.498039215686275,0.0549019607843137}
            \definecolor{color2}{rgb}{0.172549019607843,0.627450980392157,0.172549019607843}
            \definecolor{color3}{rgb}{0.83921568627451,0.152941176470588,0.156862745098039}
            \definecolor{color4}{rgb}{0.580392156862745,0.403921568627451,0.741176470588235}
            \definecolor{color5}{rgb}{0.549019607843137,0.337254901960784,0.294117647058824}
            \definecolor{color6}{rgb}{0.890196078431372,0.466666666666667,0.76078431372549}
            \begin{flushleft}
                {%
                \setlength{\fboxrule}{1pt}
                \fcolorbox{color0}{white}{Spessore $[\SI{}{\metre}]$} 
                \fcolorbox{color1}{white}{Resistenza $[\SI{}{\metre\squared\kelvin\per\watt}]$} 
                \fcolorbox{color2}{white}{Massa superficiale $[\times\SI{.e2}{\kilo\gram\per\metre\squared}]$} \\
                \fcolorbox{color3}{white}{Trasmittanza termica periodica $Y_{12} [\SI{}{\watt\per\metre\squared\per\kelvin}]$} 
                \fcolorbox{color4}{white}{Sfasamento $[\SI{}{\hour}]$} \\
                \fcolorbox{color5}{white}{Fattore di attenuazione $[\times\SI{.e-1}{}]$} \\
                \fcolorbox{color6}{white}{Capacità termica periodica interna $k_1 [\times \SI{.e-1}{\kilo\joule\per\metre\squared\per\kelvin}]$} 
                }
            \end{flushleft}
        \end{textblock*}
\end{frame}
\begin{frame}
    \begin{block}{Problemi aperti}
        Un numero pari $>2$ è sempre la somma di due primi?
    \end{block}
    \begin{exampleblock}{Un esempio}
        $2$ è un numero primo
    \end{exampleblock}
    \begin{alertblock}{Un errore}
        $0=1$
    \end{alertblock}
\end{frame}






\end{document}