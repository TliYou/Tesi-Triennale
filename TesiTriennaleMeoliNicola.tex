\documentclass[a4paper,10pt]{report}
\usepackage{geometry}\geometry{a4paper,top=3.5cm,bottom=3.5cm,%
left=2.5cm,right=2.5cm,heightrounded,bindingoffset=0mm}
\usepackage[T1]{fontenc}
\usepackage[utf8]{inputenc}
\usepackage[italian]{babel}
\usepackage{graphicx}
\usepackage[export]{adjustbox}%Per il Frame attrono le immagini e il valign
\usepackage{subfig}
\usepackage{amsmath,amsfonts,amssymb,braket,mathrsfs}
\usepackage{float}
\usepackage{tabularx,booktabs}
\usepackage{hyperref}
\usepackage{epsfig}
\usepackage{pdfpages} %Per gli allegati
%\usepackage{minipage}
\usepackage[output-decimal-marker={,}]{siunitx}
\usepackage{tikz}
\usepackage{pgfplots,pgfplotstable}
%\pgfplotsset{compat=1.15} %indica la versione da utilizzare per pgfplot
\usetikzlibrary{patterns} % per il tratteggio
\usepgfplotslibrary{groupplots}
\pgfplotsset{compat=newest}
%\usepackage{stanli}
\usepackage{xspace}% per lo spazio intelligente
\newcommand{\e}{\`E\xspace}  %E'
\usepackage{titlesec} % per formato custom dei titoli dei capitoli
%\usepackage{sideways}%%%
% redefinizione del formato del titolo del capitolo
      % da formato
      %   Capitolo X
      %   Titolo capitolo
      %   a formato
      %      Titolo capitolo 
	\titleformat{\chapter}
        {\normalfont\Huge\bfseries}{}{0em}{}
	\titlespacing*{\chapter}{0pt}{0in}{0.02in}
	\titlespacing*{\section}{0pt}{0.2in}{0.02in}
	\titlespacing*{\subsection}{0pt}{0.10in}{0.02in}
%serve per la didascalia di tabelle e figure:
\usepackage{caption}
\captionsetup{tableposition=top,figureposition=bottom,font=small}\captionsetup{format=hang,labelfont={bf,color=pantone186}} %didascalie a più righe allineate e il nome in grassetto
%non viene allineato a sinistra se la didascalia è corta una sola riga. PERCHé??
\usepackage{xcolor}
%serve per mettere il codice con lo sfondo grigio chiaro
\definecolor{pantone186}{RGB}{206, 17, 38} %il colore del logo UNITN
\definecolor{myGray}{gray}{0.5} %più basso più scuro è
\usepackage{listings} 
\lstset{basicstyle=\scriptsize\ttfamily,
backgroundcolor=\color{lightgray},%
boxpos=c,%
stringstyle=\itshape,		
lineskip=3pt,%
numbers=left,
numberstyle=\tiny,}
\usepackage{lscape}
\usepackage{multirow}
\usepackage{import}
%\usepackage{pythontex}
\begin{document}
\input{CHAPTERS/Titolo.tex}
\tableofcontents
%\setcounter{page}{1}
%Tabelle e figure sulla stessa pagina:
%Le aggiunge all'indice. phantomsection serve per non far casini con hyperref
\clearpage
\begingroup
   %\let\cleardoublepage\relax  % book
    \let\clearpage\relax        % report
        \listoftables
        \phantomsection
        \addcontentsline{toc}{chapter}{Elenco delle tabelle}
        %
        \listoffigures
        \phantomsection
        \addcontentsline{toc}{chapter}{Elenco delle figure}
\endgroup
%

%!TEX root = ../TesiTriennaleMeoliNicola.tex
\chapter{Introduzione}
Sappiamo come nell'ambito della progettazione edilizia esistano un'infinità di soluzioni costruttive,ma quel è la migliore tra tutte? 
In un precedente lavoro di gruppo era stata analizzata la convenienza economica di diversi pacchetti costruttivi di parete, prendendo come parametri di confronto il costo iniziale, di manutenzione e il tempo di messa in opera. 
Per avere soluzioni simili era stato scelto come parametro fisso la resistenza termica ponendola pari a $R=\SI{4}{m^2K\per W}$.
Il confronto era tra pacchetti costituiti da materiali ben differenti tra loro; tra cui telaio in calcestruzzo armato, telaio in acciaio, muratura portante e parete in X-LAM portante.
A questi strati strutturali si erano poi aggiunti gli isolanti e le rifiniture interne ed esterne.

Per quanto riguarda lo strato strutturale si era arrivati alla conclusione che, dal punto di vista economico, conveniva di più la soluzione in calcelstruzzo armato. 
Perché sebbene richieda molto più tempo di posa in opera, essa è anche molto meno costosa. 
Al lato opposto si aveva la soluzione in X-LAM. 
Ovvero l'unica proposta scelta che prevedeva l'intera parete con una tipologia costruttiva prefabbricata.
Proprio per questo risultava la più rapida, e ciò permetteva di avere una durata del cantiere molto più breve rispetto le altre. 
In più richiedeva meno manutenzione.
Questo era un vantaggio economico nel lungo periodo, ma uno svantaggio enorme iniziale perché il costo di costruzione a causa solo del materiale, era molto più elevato.

(GRAFICO FINALE COSTI O DATI CON I COSTI)

Ora invece si vogliono prendere in esame soltanto due pacchetti e confrontarli non più dal punto di vista economico, ma da quello termo-igrometrico.
Si vuole cercare di capire, cioè, se nonostante i costi più elevati iniziali, ci siano dei vantaggi che facciano scegliere per il pacchetto composto da legno.

Nei capitoli successivi si entrerà più nel dettaglio, riportando un'analisi di Glaser effettuata e una ricerca accademica su questo materiale diametralmente opposto a tutti gli altri.

(DA AGGIUNGERE QUALCHE DATO DELLA VECCHIA RELAZIONE)
%!TEX root = ../TesiTriennaleMeoliNicola.tex
\chapter{Analisi statica}
\clearpage
\begin{landscape}
        \subsection*{Parete solo mattone pieno (0a)}
                \begin{minipage}[c]{0.3\linewidth}
                        \input{CHAPTERS/tabella-input-0a.tex}
                \end{minipage}
                \hspace*{0.15\linewidth}
                \begin{minipage}[c]{0.55\linewidth}
                        \input{CHAPTERS/temperatura+pressione--parete-0a.tex}    
                \end{minipage}
        \subsection*{Parete solo isolante (0b)}
                \begin{minipage}[c]{0.3\linewidth}
                        \begin{table}[H]
\centering
\begin{tabular}{lS[table-format=1.3]S[table-format=1.3]S[table-format=2.1]}
\toprule
\multicolumn{1}{c}{\multirow{3}{*}{Strati}} & \multicolumn{1}{c}{Spessori} & \multicolumn{1}{c}{Conducibilità} & \multicolumn{1}{c}{Permeabilità} \\
\multicolumn{1}{c}{} & \multicolumn{1}{c}{$s$} & \multicolumn{1}{c}{termica $\lambda$} & \multicolumn{1}{c}{al vapore $\mu$} \\
\multicolumn{1}{c}{} & \multicolumn{1}{c}{[\SI{}{\metre}]} & \multicolumn{1}{c}{[\SI{}{\watt\per\metre\squared\per\kelvin}]} & \multicolumn{1}{c}{$[-]$} \\
\midrule
 Isolante VentirockDuo  &     0,12 &         0,035 &  1,0 \\
\bottomrule
\end{tabular}
\end{table}

                \end{minipage}
                \hspace*{0.15\linewidth}
                \begin{minipage}[c]{0.55\linewidth}
                        \input{CHAPTERS/temperatura+pressione--parete-0b.tex}    
                \end{minipage}
        \subsection*{Parete in laterizio con isolante interno (1a)}
                \begin{minipage}[c]{0.3\linewidth}
                        \begin{tabular}{lS[table-format=1.3]S[table-format=1.3]S[table-format=2.1]}
\toprule
\multicolumn{1}{c}{\multirow{3}{*}{Strati}} & \multicolumn{1}{c}{Spessori} & \multicolumn{1}{c}{Conducibilità} & \multicolumn{1}{c}{Permeabilità} \\
\multicolumn{1}{c}{} & \multicolumn{1}{c}{$s$} & \multicolumn{1}{c}{termica $\lambda$} & \multicolumn{1}{c}{al vapore $\mu$} \\
\multicolumn{1}{c}{} & \multicolumn{1}{c}{[\SI{}{\metre}]} & \multicolumn{1}{c}{[\SI{}{\watt\per\metre\squared\per\kelvin}]} & \multicolumn{1}{c}{$[-]$} \\
\midrule
              Intonaco &    0,015 &         0,600 &   6,0 \\
 Isolante VentirockDuo &    0,120 &         0,035 &   1,0 \\
   Laterizio semipieno &    0,200 &         0,530 &  10,0 \\
              Intonaco &    0,015 &         0,900 &   8,0 \\
\bottomrule
\end{tabular}

                \end{minipage}
                \hspace*{0.15\linewidth}
                \begin{minipage}[c]{0.55\linewidth}
                        % This file was created by tikzplotlib v0.9.4.
\begin{tikzpicture}

\definecolor{color0}{rgb}{1,0.647058823529412,0}

\begin{groupplot}[group style={group size=1 by 2,vertical sep=2.5cm}]
\nextgroupplot[
	 ticklabel style={ 
 		 /pgf/number format/fixed, 
 		 /pgf/number format/precision=5
		}, 
scaled ticks=false,
height=7cm,
legend cell align={left},
legend style={fill opacity=0.8, draw opacity=1, text opacity=1, draw=white!80!black},
minor xtick={},
minor ytick={},
tick align=outside,
tick pos=left,
width=\linewidth,
x grid style={white!69.0196078431373!black},
xlabel={Spessore parete (m)},
xmajorgrids,
xmin=-0.0185, xmax=0.3885,
xtick style={color=black},
xtick={0,0.015,0.155,0.355,0.37},
xticklabel style = {rotate=90.0},
y grid style={white!69.0196078431373!black},
ylabel={Temperature (°C)},
ymin=-6.01137676302392, ymax=20.5685707562642,
ytick style={color=black},
ytick={-10,-5,0,5,10,15,20,25}
]
\addplot [line width=1.5pt, green!50.1960784313725!black]
table {%
0 19.3603913235693
0.015 19.2373896550249
0.155 -0.442877312074838
0.355 -4.72119621796609
0.37 -4.803197330329
};
\addlegendentry{Temperatura}

\nextgroupplot[
	 ticklabel style={ 
 		 /pgf/number format/fixed, 
 		 /pgf/number format/precision=5
		}, 
scaled ticks=false,
height=7cm,
legend cell align={left},
legend style={fill opacity=0.8, draw opacity=1, text opacity=1, draw=white!80!black},
minor xtick={},
minor ytick={},
tick align=outside,
tick pos=left,
width=\linewidth,
x grid style={white!69.0196078431373!black},
xlabel={Spessore equivalente Sd (m)},
xmajorgrids,
xmin=-0.1175, xmax=2.4675,
xtick style={color=black},
xtick={0,0.09,0.23,2.23,2.35},
xticklabel style = {rotate=90.0},
y grid style={white!69.0196078431373!black},
ylabel={Pressione (Pa)},
ymin=266.815315862144, ymax=2340.26179765774,
ytick style={color=black},
ytick={250,500,750,1000,1250,1500,1750,2000,2250,2500}
]
\addplot [line width=1.5pt, color0]
table {%
0 1519.01824347152
0.09 1474.67101690856
0.23 1405.68644225507
2.23 420.192518633768
2.35 361.06288321649
};
\addlegendentry{Pressione}
\addplot [line width=1.5pt, red]
table {%
0 2246.0142303034
0.09 2228.8857975183
0.23 588.588856594798
2.23 410.858360238758
2.35 407.990247749789
};
\addlegendentry{Pressione Saturazione}
\end{groupplot}

\end{tikzpicture}
    
                \end{minipage}
        \subsection*{Parete in laterizio con isolante esterno (1b)}
                \begin{minipage}[c]{0.3\linewidth}
                        \begin{table}[H]
\centering
\caption{Parete in muratura con isolante esterno}
\begin{tabular}{lrrr}
\toprule
            Strati & Spessori & Conduttività  &    mu \\
\midrule
          Intonaco &    0,015 &                0,600 &   6,0 \\
 Laterizio Poroton &    0,200 &                0,230 &  10,0 \\
  Isolante RockDuo &    0,140 &                0,035 &   1,0 \\
          Intonaco &    0,015 &                0,900 &   8,0 \\
\bottomrule
\end{tabular}
\end{table}

                \end{minipage}
                \hspace*{0.15\linewidth}
                \begin{minipage}[c]{0.55\linewidth}
                        \input{CHAPTERS/temperatura+pressione--parete-1b.tex}    
                \end{minipage}
        \subsection*{Parete in muratura Poroton con isolante esterno (2a)}
                \begin{minipage}[c]{0.3\linewidth}
                        \input{CHAPTERS/tabella-input-2a.tex}
                \end{minipage}
                \hspace*{0.15\linewidth}
                \begin{minipage}[c]{0.55\linewidth}
                        \input{CHAPTERS/temperatura+pressione--parete-2a.tex}    
                \end{minipage}
        \subsection*{Parete in X-LAM con isolante bassa densità (3a)}
                \begin{minipage}[c]{0.3\linewidth}
                        \begin{table}[H]
\centering
\caption{Perete in X-LAM con isolante bassa densità}
\begin{tabular}{lrrr}
\toprule
                      Strati & Spessori & Conduttività  &    mu \\
\midrule
                  Gessofibra &    0,013 &                0,210 &   5,0 \\
                   X-LAM KLH &    0,096 &                0,130 &  25,0 \\
 Isolante bassa densità  &    0,120 &                0,038 &   1,0 \\
              Intonaco calce &    0,015 &                0,900 &  20,0 \\
\bottomrule
\end{tabular}
\end{table}

                \end{minipage}
                \hspace*{0.15\linewidth}
                \begin{minipage}[c]{0.55\linewidth}
                        % This file was created by tikzplotlib v0.9.4.
\begin{tikzpicture}

\definecolor{color0}{rgb}{1,0.647058823529412,0}

\begin{groupplot}[group style={group size=1 by 2,vertical sep=2.5cm}]
\nextgroupplot[
	 ticklabel style={ 
 		 /pgf/number format/fixed, 
 		 /pgf/number format/precision=5
		}, 
scaled ticks=false,
height=7cm,
legend cell align={left},
legend style={fill opacity=0.8, draw opacity=1, text opacity=1, draw=white!80!black},
minor xtick={},
minor ytick={},
tick align=outside,
tick pos=left,
width=\linewidth,
x grid style={white!69.0196078431373!black},
xlabel={Spessore parete (m)},
xmajorgrids,
xmin=-0.0122, xmax=0.2562,
xtick style={color=black},
xtick={0,0.013,0.109,0.229,0.244},
xticklabel style = {rotate=90.0},
y grid style={white!69.0196078431373!black},
ylabel={Temperature (°C)},
ymin=-5.95747378805511, ymax=20.414641642178,
ytick style={color=black},
ytick={-10,-5,0,5,10,15,20,25}
]
\addplot [line width=1.5pt, green!50.1960784313725!black]
table {%
0 19.2159091226219
0.013 18.8425325143467
0.109 14.3885251635601
0.229 -4.65821679704034
0.244 -4.75874126849906
};
\addlegendentry{Temperatura}

\nextgroupplot[
	 ticklabel style={ 
 		 /pgf/number format/fixed, 
 		 /pgf/number format/precision=5
		}, 
scaled ticks=false,
height=7cm,
legend cell align={left},
legend style={fill opacity=0.8, draw opacity=1, text opacity=1, draw=white!80!black},
minor xtick={},
minor ytick={},
tick align=outside,
tick pos=left,
width=\linewidth,
x grid style={white!69.0196078431373!black},
xlabel={Spessore equivalente Sd (m)},
xmajorgrids,
xmin=-0.14425, xmax=3.02925,
xtick style={color=black},
xtick={0,0.065,2.465,2.585,2.885},
xticklabel style = {rotate=90.0},
y grid style={white!69.0196078431373!black},
ylabel={Pressione (Pa)},
ymin=267.820712152638, ymax=2319.14847555738,
ytick style={color=black},
ytick={250,500,750,1000,1250,1500,1750,2000,2250,2500}
]
\addplot [line width=1.5pt, color0]
table {%
0 1519.01824347152
0.065 1492.92912790252
2.465 529.638706893132
2.585 481.474185842663
2.885 361.06288321649
};
\addlegendentry{Pressione}
\addplot [line width=1.5pt, red]
table {%
0 2225.90630449353
0.065 2174.67021383189
2.465 1638.46020552403
2.585 413.073607008036
2.885 409.542895587558
};
\addlegendentry{Pressione Saturazione}
\end{groupplot}

\end{tikzpicture}
    
                \end{minipage}
        \subsection*{Parete in X-LAM con isolante bassa densità lana di roccia (3b)}
                \begin{minipage}[c]{0.3\linewidth}
                        \begin{table}[H]
\centering
\caption{Perete in X-LAM con isolante bassa densità lana di roccia}
\begin{tabular}{lrrr}
\toprule
                Strati & Spessori & Conduttività &    mu \\
\midrule
            Gessofibra &    0,013 &                0,210 &   5,0 \\
             X-LAM KLH &    0,096 &                0,130 &  25,0 \\
 Isolante ventirockduo &    0,105 &                0,035 &   1,0 \\
        Intonaco calce &    0,015 &                0,900 &  20,0 \\
\bottomrule
\end{tabular}
\end{table}

                \end{minipage}
                \hspace*{0.15\linewidth}
                \begin{minipage}[c]{0.55\linewidth}
                        % This file was created by tikzplotlib v0.9.4.
\begin{tikzpicture}

\definecolor{color0}{rgb}{1,0.647058823529412,0}

\begin{groupplot}[group style={group size=1 by 2,vertical sep=2.5cm}]
\nextgroupplot[
	 ticklabel style={ 
 		 /pgf/number format/fixed, 
 		 /pgf/number format/precision=5
		}, 
scaled ticks=false,
height=7cm,
legend cell align={left},
legend style={fill opacity=0.8, draw opacity=1, text opacity=1, draw=white!80!black},
minor xtick={},
minor ytick={},
tick align=outside,
tick pos=left,
width=\linewidth,
x grid style={white!69.0196078431373!black},
xlabel={Spessore parete (m)},
xmajorgrids,
xmin=-0.01145, xmax=0.24045,
xtick style={color=black},
xtick={0,0.013,0.109,0.214,0.229},
xticklabel style = {rotate=90.0},
y grid style={white!69.0196078431373!black},
ylabel={Temperature (°C)},
ymin=-5.94588914613306, ymax=20.3815597265862,
ytick style={color=black},
ytick={-10,-5,0,5,10,15,20,25}
]
\addplot [line width=1.5pt, green!50.1960784313725!black]
table {%
0 19.1848575050989
0.013 18.7966944122889
0.109 14.1662991749812
0.214 -4.64468147658159
0.229 -4.74918692464583
};
\addlegendentry{Temperatura}

\nextgroupplot[
	 ticklabel style={ 
 		 /pgf/number format/fixed, 
 		 /pgf/number format/precision=5
		}, 
scaled ticks=false,
height=7cm,
legend cell align={left},
legend style={fill opacity=0.8, draw opacity=1, text opacity=1, draw=white!80!black},
minor xtick={},
minor ytick={},
tick align=outside,
tick pos=left,
width=\linewidth,
x grid style={white!69.0196078431373!black},
xlabel={Spessore equivalente Sd (m)},
xmajorgrids,
xmin=-0.1435, xmax=3.0135,
xtick style={color=black},
xtick={0,0.065,2.465,2.57,2.87},
xticklabel style = {rotate=90.0},
y grid style={white!69.0196078431373!black},
ylabel={Pressione (Pa)},
ymin=268.035757521022, ymax=2314.63252282132,
ytick style={color=black},
ytick={250,500,750,1000,1250,1500,1750,2000,2250,2500}
]
\addplot [line width=1.5pt, color0]
table {%
0 1519.01824347152
0.065 1492.79277363996
2.465 524.46773370544
2.57 482.103513208305
2.87 361.06288321649
};
\addlegendentry{Pressione}
\addplot [line width=1.5pt, red]
table {%
0 2221.60539712585
0.065 2168.45182859209
2.465 1615.05395862344
2.57 413.551117171863
2.87 409.87728665048
};
\addlegendentry{Pressione Saturazione}
\end{groupplot}

\end{tikzpicture}
    
                \end{minipage}
        \subsection*{Parete in X-LAM con isolante alta densità fibra di legno (3c)}
                \begin{minipage}[c]{0.3\linewidth}
                        \begin{table}[H]
\centering
\caption{Perete in X-LAM con isolante alta densità fibra di legno}
\begin{tabular}{lrrr}
\toprule
                                    Strati & Spessori & Conduttività  &    mu \\
\midrule
                                Gessofibra &    0,013 &                0,210 &   5,0 \\
                                 X-LAM KLH &    0,096 &                0,130 &  25,0 \\
 Isolante alta densità &    0,130 &                0,043 &   5,0 \\
                            Intonaco calce &    0,015 &                0,900 &  20,0 \\
\bottomrule
\end{tabular}
\end{table}

                \end{minipage}
                \hspace*{0.15\linewidth}
                \begin{minipage}[c]{0.55\linewidth}
                        % This file was created by tikzplotlib v0.9.4.
\begin{tikzpicture}

\definecolor{color0}{rgb}{1,0.647058823529412,0}

\begin{groupplot}[group style={group size=1 by 2,vertical sep=2.5cm}]
\nextgroupplot[
	 ticklabel style={ 
 		 /pgf/number format/fixed, 
 		 /pgf/number format/precision=5
		}, 
scaled ticks=false,
height=7cm,
legend cell align={left},
legend style={fill opacity=0.8, draw opacity=1, text opacity=1, draw=white!80!black},
minor xtick={},
minor ytick={},
tick align=outside,
tick pos=left,
width=\linewidth,
x grid style={white!69.0196078431373!black},
xlabel={Spessore parete (m)},
xmajorgrids,
xmin=-0.0127, xmax=0.2667,
xtick style={color=black},
xtick={0,0.013,0.109,0.239,0.254},
xticklabel style = {rotate=90.0},
y grid style={white!69.0196078431373!black},
ylabel={Temperature (°C)},
ymin=-5.94765269629741, ymax=20.3865958440658,
ytick style={color=black},
ytick={-10,-5,0,5,10,15,20,25}
]
\addplot [line width=1.5pt, green!50.1960784313725!black]
table {%
0 19.1895845467766
0.013 18.803672426194
0.109 14.2001290232679
0.239 -4.64674198192825
0.254 -4.75064139900817
};
\addlegendentry{Temperatura}

\nextgroupplot[
	 ticklabel style={ 
 		 /pgf/number format/fixed, 
 		 /pgf/number format/precision=5
		}, 
scaled ticks=false,
height=7cm,
legend cell align={left},
legend style={fill opacity=0.8, draw opacity=1, text opacity=1, draw=white!80!black},
minor xtick={},
minor ytick={},
tick align=outside,
tick pos=left,
width=\linewidth,
x grid style={white!69.0196078431373!black},
xlabel={Spessore equivalente Sd (m)},
xmajorgrids,
xmin=-0.17075, xmax=3.58575,
xtick style={color=black},
xtick={0,0.065,2.465,3.115,3.415},
xticklabel style = {rotate=90.0},
y grid style={white!69.0196078431373!black},
ylabel={Pressione (Pa)},
ymin=268.003044265693, ymax=2315.31950118321,
ytick style={color=black},
ytick={250,500,750,1000,1250,1500,1750,2000,2250,2500}
]
\addplot [line width=1.5pt, color0]
table {%
0 1519.01824347152
0.065 1496.97809752231
2.465 683.188093243512
3.115 462.786633751339
3.415 361.06288321649
};
\addlegendentry{Pressione}
\addplot [line width=1.5pt, red]
table {%
0 2222.25966223242
0.065 2169.39745901465
2.465 1618.59809228845
3.115 413.4783925652
3.415 409.826365687118
};
\addlegendentry{Pressione Saturazione}
\end{groupplot}

\end{tikzpicture}
    
                \end{minipage}
                \clearpage       
\end{landscape}
\clearpage
%!TEX root = ../TesiTriennaleMeoliNicola.tex
\chapter{Analisi dinamica}
\section*{Alcune definizioni e nomenclature}
Con la denominazione dell'á'pice $^{(k)}$ d'ora in avanti si farà riferimento ad una quantità che và calcolata per ogni strato $k$-esimo. 
Con il  grassetto verranno indicate delle matrici, mentre con un doppio pedice le componenti di tali matrici.
\begin{table}[H]
\centering
\begin{tabular}{p{0.3\textwidth}p{0.7\textwidth}}
    \toprule
    \textbf{Profondità di penetrazione}  \[\delta^{(k)} = \sqrt{\frac{\lambda^{(k)} \, T}{\pi \,\rho^{(k)} \,c^{(k)}}} \,\left[\SI{}{\metre}\right]\] & definizione definizione definizione definizione definizione definizione definizione definizione definizione definizione definizione definizione definizione definizione definizione \\
    \textbf{Rapporto lunghezze} \[\xi^{(k)} = \dfrac{s^{(k)}}{\delta^{(k)}} \,\left[\SI{}{-}\right] \] & definizione definizione definizione definizione definizione definizione definizione definizione definizione definizione definizione definizione definizione definizione definizione \\
    \bottomrule
\end{tabular}
\end{table}
\section*{Metodologia secondo la normativa}
La matrice di trasferimento, calcolata per ogni singolo strato $k$-esimo è definita come:
\begin{equation}
    \mathbf{Z}^{(k)} =
    \begin{bmatrix}  
        Z_{11}^{(k)} & Z_{12}^{(k)} \\
        Z_{21}^{(k)} & Z_{22}^{(k)}
    \end{bmatrix}
\end{equation}
dove le componenti della matrice $\mathbf{Z^{(k)}}$ sono

{\footnotesize
\begin{align*}
    Z_{11}^{(k)} &= Z_{22}^{(k)} = \cosh\left(\xi^{(k)}\right)\cos\left(\xi^{(k)}\right) + \mathrm{i} \left[\sinh\left(\xi^{(k)}\right)\sin\left(\xi^{(k)}\right)\right] \\
    Z_{12}^{(k)} &=-\frac{\delta^{(k)}}{2\lambda^{(k)}}\Bigl\{\sinh\left(\xi^{(k)}\right)\cos\left(\xi^{(k)}\right)+\cosh\left(\xi^{(k)}\right)\sin\left(\xi^{(k)}\right) + \mathrm{i}\left[\cosh\left(\xi^{(k)}\right)\sin\left(\xi^{(k)}\right)-\sinh\left(\xi^{(k)}\right)\cos\left(\xi^{(k)}\right)\right]  \Bigr\} \\
    Z_{21}^{(k)} &= \frac{\lambda^{(k)}}{\delta^{(k)}}\Bigl\{  \sinh\left(\xi^{(k)}\right)\cos\left(\xi^{(k)}\right)-\cosh\left(\xi^{(k)}\right)\sin\left(\xi^{(k)}\right) + \mathrm{i} \left[\sinh\left(\xi^{(k)}\right)\cos\left(\xi^{(k)}\right)+\cosh\left(\xi^{(k)}\right)\sin\left(\xi^{(k)}\right)\right] \Bigr\}
\end{align*}
}

Moltiplicando le matrici di trasferimento di ogni singolo strato si ottiene la matrice di trasferimento dell'intero componente edilizio relativa agli $N$  strati:
\begin{equation}
    \mathbf{Z}=\mathbf{Z}_{N}\cdot\mathbf{Z}_{N-1}\cdot\ldots\cdot\mathbf{Z}_{2}\cdot\mathbf{Z}_{1} \qquad .
\end{equation}
Calcolando le matrici di trasferimento degli strati liminari come
\begin{equation}
    \mathbf{Z}_{\text{int.}} = 
    \begin{bmatrix}
        1 & -R_{\text{s,int.}} \\
        0 & 1
    \end{bmatrix}
    \qquad
    \mathbf{Z}_{\text{est.}} = 
    \begin{bmatrix}
        1 & -R_{\text{s,est.}} \\
        0 & 1
    \end{bmatrix}
\end{equation}
si può infine ottenere la matrice di trasferimento da ambiente ad ambiente:
\begin{equation}
    \mathbf{Z}_{\text{ee}} = \mathbf{Z}_{\text{est.}} \cdot \mathbf{Z} \cdot \mathbf{Z}_{\text{int.}} \qquad .
\end{equation}

Dagli elementi di questa matrice è possibile ricavare le grandezze di nostro interesse per l'analisi dinamica.
Dall'elemento {\footnotesize $12$} della matrice $\mathbf{Z_{\text{ee}}}$ si ricava la trasmittanza termica periodica $Y_{12}$ tramite il modulo della parte reale e immaginaria
\begin{equation}
    Y_{12} = \left\lvert -\frac{1}{Z_{\text{ee},12}}\right\rvert 
\end{equation} 
e il fattore di attenuazione $f$
\begin{equation}
    f = -\frac{Y_{12}}{U} \qquad .
\end{equation}
Lo sfasamento riferito al tempo iniziale si ottiene infine come:
\begin{equation}
    \Delta T = \frac{T}{2\,\pi}\arg\left(Z_{\text{ee},12}\right) + \frac{T}{2} \qquad .
\end{equation}
Dagli elementi {\footnotesize $11$} e {\footnotesize $22$} si ricavano riispettivamente l'ammettenza termica interna ed esterna 
\begin{equation}
    Y_{\text{ii}}=Y_{11} = \left\lvert -\frac{1}{Z_{\text{ee},11}}\right\rvert \qquad \qquad
    Y_{\text{ee}}=Y_{22} = \left\lvert -\frac{1}{Z_{\text{ee},22}}\right\rvert \qquad .
\end{equation} 
Un'altra quantità molto utile è la capacità termica periodica interna per unità di superficie. \e definita dalla normativa come $k_1 = \frac{T}{2\,\pi} \left\lvert Y_{11} - Y_{12}\right\rvert$ e sostituendo i termini in funzione dalla matrice di trasfermento, si può riscrivere come:
\begin{equation}
    k_1 = \frac{T}{2\,\pi} \left\lvert \frac{Z_{\text{ee},11} - 1}{Z_{\text{ee},12}}\right\rvert\,.
\end{equation}

\begin{landscape}
        \clearpage
        \subsection*{Parete solo mattone pieno (0a)}
        \input{CHAPTERS/tabella-dinamica-0a.tex}
        \clearpage
        \subsection*{Parete solo isolante (0b)}
        \input{CHAPTERS/tabella-dinamica-0b.tex}
        \clearpage
        \subsection*{Parete in laterizio con isolante interno (1a)}
        \begin{table}[H]
\centering
\resizebox{\linewidth}{!}{%
\begin{tabular}{@{}
				l
				S[table-format=1.3]
				S[table-format=4.1]
				S[table-format=4.1]
				S[table-format=2.2]
				S[table-format=1.3]
				S[table-format=1.3]
				S[table-format=2.1]
				S[table-format=1.1]
				S[table-format=4.1]
				@{}
				}
\toprule
\multicolumn{1}{c}{\multirow{3}{*}{Strati}} & \multicolumn{1}{c}{\multirow{2}{*}{Spessori}}    & \multicolumn{1}{c}{\multirow{2}{*}{Densità}}                                            & \multicolumn{1}{c}{Calore}                                            & \multicolumn{1}{c}{Massa}                                               & \multicolumn{1}{c}{Profondità di}                      & \multicolumn{1}{c}{Rapporto}   & \multicolumn{1}{c}{Capacità}                                                  & \multicolumn{1}{c}{Diffusività}                                      & \multicolumn{1}{c}{Effusività}                                                             \\
\multicolumn{1}{c}{}                        & \multicolumn{1}{c}{}                              & \multicolumn{1}{c}{}                                                  & \multicolumn{1}{c}{specifico}                                         & \multicolumn{1}{c}{superficiale}                                        & \multicolumn{1}{c}{penetrazione $\delta$} & \multicolumn{1}{c}{$\xi$}      & \multicolumn{1}{c}{termica areica}                                            & \multicolumn{1}{c}{termica}                                          & \multicolumn{1}{c}{Termica}                                                                \\
\multicolumn{1}{c}{}                        & \multicolumn{1}{c}{$\left[\SI{}{\metre}\right]$} & \multicolumn{1}{c}{$\left[\SI{}{\kilo\gram\per\metre\cubed}\right]$} & \multicolumn{1}{c}{$\left[\SI{}{\joule\per\kilo\gram\per\kelvin}\right]$} & \multicolumn{1}{c}{$\left[\SI{}{\kilo\gram\per\metre\squared}\right]$} & \multicolumn{1}{c}{$\left[\SI{}{\metre}\right]$}      & \multicolumn{1}{c}{$[-]$} & \multicolumn{1}{c}{$\left[\SI{}{\kilo\joule\per\metre\squared\per\kelvin}\right]$} & \multicolumn{1}{c}{$\left[\SI{.e-7}{\metre\squared\per\second}\right]$} & \multicolumn{1}{c}{$\left[\SI{}{\watt\second\tothe{0.5}\per\metre\squared\per\kelvin}\right]$} \\
\midrule
   Intonaco              & 0,015 & 1500,0 & 1000,0 & 22,5  & 0,105 & 0,143 & 22,5  & 4,00 & 948,7 \\
   Isolante VentirockDuo & 0,120 & 70,0   & 1030,0 & 8,4   & 0,116 & 1,039 & 8,7   & 4,85 & 50,2 \\
   Laterizio semipieno   & 0,200 & 1000,0 & 840,0  & 200,0 & 0,132 & 1,518 & 168,0 & 6,31 & 667,2 \\
   Intonaco              & 0,015 & 1800,0 & 1000,0 & 27,0  & 0,117 & 0,128 & 27,0  & 5,00 & 1272,8 \\
\bottomrule
\end{tabular}%
}
\end{table}

\begin{flushleft}
\begin{align*}
\text{Massa superficiale totale} \, M_s &= \SI{257.9}{\kilo\gram\per\metre\squared}\\
\text{Sfasamento} \, \Delta\tau &= \SI{9.09}{\hour}\\
\text{Fattore di attenuazione} \, fd &= \SI{0.352}{}\\
\text{Trasmittanza termica periodica} \, Y_{12} &= \SI{0.088}{\watt\per\metre\squared\per\kelvin}\\
\text{Ammettanza termica interna} \, Y_{11} &= \SI{1.744}{\watt\per\metre\squared\per\kelvin}\\
\text{Ammettanza termica esterna} \, Y_{22} &= \SI{5.991}{\watt\per\metre\squared\per\kelvin}\\
\text{Capacità termica periodica interna} \, k_1 &= \SI{25.09}{\kilo\joule\per\metre\squared\per\kelvin}\\
\end{align*}
\end{flushleft}

        \clearpage
        \subsection*{Parete in laterizio con isolante esterno (1b)}
        \begin{table}
\centering
\caption{Parete in muratura con isolante esterno}
\begin{tabular}{lrrrrrr}
\toprule
            Strati & Spessori & Densità & Calore specifico & Massa superficiale & Profondità di Penetrazione &     xi \\
\midrule
          Intonaco &    0,015 &  1500,0 &           1000,0 &               22,5 &                      0,105 &  0,143 \\
 Laterizio Poroton &    0,200 &   860,0 &            840,0 &              172,0 &                      0,094 &  2,137 \\
  Isolante RockDuo &    0,140 &  1200,0 &           1500,0 &              168,0 &                      0,023 &  6,054 \\
          Intonaco &    0,015 &  1800,0 &           1000,0 &               27,0 &                      0,117 &  0,128 \\
\bottomrule
\end{tabular}
\end{table}

\begin{flushleft}
\begin{align*}
\text{Massa superficiale totale} &= \SI{389.5}{\kilo\gram}\\
\text{Sfasamento} &= \SI{6.55}{\hour}\\
\text{Attenuazione} &= \SI{304.872}{}
\end{align*}
\end{flushleft}

        \clearpage
        \subsection*{Parete in muratura Poroton con isolante esterno (2a)}
        \begin{table}[H]
\centering
\resizebox{\linewidth}{!}{%
\begin{tabular}{@{}
				l
				S[table-format=1.3]
				S[table-format=4.1]
				S[table-format=4.1]
				S[table-format=2.2]
				S[table-format=1.3]
				S[table-format=1.3]
				S[table-format=2.1]
				S[table-format=1.1]
				S[table-format=4.1]
				@{}
				}
\toprule
\multicolumn{1}{c}{\multirow{3}{*}{Strati}} & \multicolumn{1}{c}{\multirow{2}{*}{Spessori}}    & \multicolumn{1}{c}{\multirow{2}{*}{Densità}}                                            & \multicolumn{1}{c}{Calore}                                            & \multicolumn{1}{c}{Massa}                                               & \multicolumn{1}{c}{Profondità di}                      & \multicolumn{1}{c}{Rapporto}   & \multicolumn{1}{c}{Capacità}                                                  & \multicolumn{1}{c}{Diffusività}                                      & \multicolumn{1}{c}{Effusività}                                                             \\
\multicolumn{1}{c}{}                        & \multicolumn{1}{c}{}                              & \multicolumn{1}{c}{}                                                  & \multicolumn{1}{c}{specifico}                                         & \multicolumn{1}{c}{superficiale}                                        & \multicolumn{1}{c}{penetrazione $\delta$} & \multicolumn{1}{c}{$\xi$}      & \multicolumn{1}{c}{termica areica}                                            & \multicolumn{1}{c}{termica}                                          & \multicolumn{1}{c}{Termica}                                                                \\
\multicolumn{1}{c}{}                        & \multicolumn{1}{c}{$\left[\SI{}{\metre}\right]$} & \multicolumn{1}{c}{$\left[\SI{}{\kilo\gram\per\metre\cubed}\right]$} & \multicolumn{1}{c}{$\left[\SI{}{\joule\per\kilo\gram\per\kelvin}\right]$} & \multicolumn{1}{c}{$\left[\SI{}{\kilo\gram\per\metre\squared}\right]$} & \multicolumn{1}{c}{$\left[\SI{}{\metre}\right]$}      & \multicolumn{1}{c}{$[-]$} & \multicolumn{1}{c}{$\left[\SI{}{\kilo\joule\per\metre\squared\per\kelvin}\right]$} & \multicolumn{1}{c}{$\left[\SI{.e-7}{\metre\squared\per\second}\right]$} & \multicolumn{1}{c}{$\left[\SI{}{\watt\second\tothe{0.5}\per\metre\squared\per\kelvin}\right]$} \\
\midrule
              Intonaco &    0,015 &  1500,0 &           1000,0 &               22,5 &                      0,105 &  0,143 &             22,5 &                       4,00 &              948,7 \\
     Laterizio Poroton &    0,200 &   860,0 &            840,0 &              172,0 &                      0,094 &  2,137 &            144,5 &                       3,18 &              407,6 \\
 Isolante VentirockDuo &    0,110 &    70,0 &           1030,0 &                7,7 &                      0,116 &  0,952 &              7,9 &                       4,85 &               50,2 \\
              Intonaco &    0,015 &  1800,0 &           1000,0 &               27,0 &                      0,117 &  0,128 &             27,0 &                       5,00 &             1272,8 \\
\bottomrule
\end{tabular}%
}
\end{table}

\begin{flushleft}
\begin{align*}
\text{Massa superficiale totale} \, M_s &= \SI{229.2}{\kilo\gram\per\metre\squared}\\
\text{Sfasamento} \, \Delta\tau &= \SI{11.25}{\hour}\\
\text{Fattore di attenuazione} \, fd &= \SI{0.182}{}\\
\text{Trasmittanza termica periodica} \, Y_{12} &= \SI{0.043}{\watt\per\metre\squared\per\kelvin}\\
\text{Ammettanza termica interna} \, Y_{11} &= \SI{3.171}{\watt\per\metre\squared\per\kelvin}\\
\text{Ammettanza termica esterna} \, Y_{22} &= \SI{2.154}{\watt\per\metre\squared\per\kelvin}\\
\text{Capacità termica periodica interna} \, k_1 &= \SI{44.15}{\kilo\joule\per\metre\squared\per\kelvin}\\
\end{align*}
\end{flushleft}

        \clearpage
        \subsection*{Parete in X-LAM con isolante bassa densità (3a)}
        \begin{table}
\centering
\caption{Perete in X-LAM con isolante bassa densità NOME}
\begin{tabular}{lrrrrrr}
\toprule
                      Strati & Spessori & Densità & Calore specifico & Massa superficiale & Profondità di Penetrazione &     xi \\
\midrule
                  Gessofibra &    0,013 &  1150,0 &           1100,0 &              14,95 &                      0,068 &  0,192 \\
                   X-LAM KLH &    0,096 &   500,0 &           1600,0 &              48,00 &                      0,067 &  1,436 \\
 Isolante bassa densità NOME &    0,120 &    50,0 &           2100,0 &               6,00 &                      0,100 &  1,203 \\
              Intonaco calce &    0,015 &  1800,0 &           1000,0 &              27,00 &                      0,117 &  0,128 \\
\bottomrule
\end{tabular}
\end{table}

\begin{flushleft}
\begin{align*}
\text{Massa superficiale totale} &= \SI{95.95}{\kilo\gram}\\
\text{Sfasamento} &= \SI{9.21}{\hour}\\
\text{Attenuazione} &= \SI{248.694}{}
\end{align*}
\end{flushleft}

        \clearpage
        \subsection*{Parete in X-LAM con isolante bassa densità lana di roccia (3b)}
        \begin{table}
\centering
\caption{Perete in X-LAM con isolante bassa densità lana di roccia}
\begin{tabular}{lrrrrrr}
\toprule
                Strati & Spessori & Densità & Calore specifico & Massa superficiale & Profondità di Penetrazione &     xi \\
\midrule
            Gessofibra &    0,013 &  1150,0 &           1100,0 &              14,95 &                      0,068 &  0,192 \\
             X-LAM KLH &    0,096 &   500,0 &           1600,0 &              48,00 &                      0,067 &  1,436 \\
 Isolante ventirockduo &    0,105 &    70,0 &           1030,0 &               7,35 &                      0,116 &  0,909 \\
        Intonaco calce &    0,015 &  1800,0 &           1000,0 &              27,00 &                      0,117 &  0,128 \\
\bottomrule
\end{tabular}
\end{table}

\begin{flushleft}
\begin{align*}
\text{Massa superficiale totale} &= \SI{97.3}{\kilo\gram}\\
\text{Sfasamento} &= \SI{8.34}{\hour}\\
\text{Attenuazione} &= \SI{239.221}{}
\end{align*}
\end{flushleft}

        \clearpage
        \subsection*{Parete in X-LAM con isolante alta densità fibra di legno (3c)}
        \begin{table}
\centering
\caption{Perete in X-LAM con isolante alta densità fibra di legno}
\begin{tabular}{lrrrrrr}
\toprule
                                    Strati & Spessori & Densità & Calore specifico & Massa superficiale & Profondità di Penetrazione &     xi \\
\midrule
                                Gessofibra &    0,013 &  1150,0 &           1100,0 &              14,95 &                      0,068 &  0,192 \\
                                 X-LAM KLH &    0,096 &   500,0 &           1600,0 &              48,00 &                      0,067 &  1,436 \\
 Isolante alta densità Naturalia Diffuterm &    0,130 &   190,0 &           2100,0 &              24,70 &                      0,054 &  2,388 \\
                            Intonaco calce &    0,015 &  1800,0 &           1000,0 &              27,00 &                      0,117 &  0,128 \\
\bottomrule
\end{tabular}
\end{table}

\begin{flushleft}
\begin{align*}
\text{Massa superficiale totale} &= \SI{114.65}{\kilo\gram}\\
\text{Sfasamento} &= \SI{13.65}{\hour}\\
\text{Attenuazione} &= \SI{240.616}{}
\end{align*}
\end{flushleft}

        %%%%%%%%%%%%%%%%%%%%%%%%%%%%%%%%%%%%%%%%%%%%%%%%%%%%%%%
        \subsection*{Confronto}
        % This file was created by tikzplotlib v0.9.4.
\begin{tikzpicture}

\definecolor{color0}{rgb}{0.12156862745098,0.466666666666667,0.705882352941177}
\definecolor{color1}{rgb}{1,0.498039215686275,0.0549019607843137}
\definecolor{color2}{rgb}{0.172549019607843,0.627450980392157,0.172549019607843}
\definecolor{color3}{rgb}{0.83921568627451,0.152941176470588,0.156862745098039}

\begin{axis}[
height=15cm,
legend cell align={left},
legend style={fill opacity=0.8, draw opacity=1, text opacity=1, at={(0.03,0.97)}, anchor=north west, draw=white!80!black},
tick align=outside,
tick pos=left,
width=\linewidth,
x grid style={white!69.0196078431373!black},
xmin=-0.629, xmax=4.629,
xtick style={color=black},
xtick={0,1,2,3,4},
xticklabels={1a,1b,3a,3b,3c},
y grid style={white!69.0196078431373!black},
ymin=0, ymax=14.3356557482948,
ytick style={color=black}
]
\draw[draw=none,fill=color0] (axis cs:-0.39,0) rectangle (axis cs:-0.21,5.08123188405797);
\draw[draw=none,fill=color0] (axis cs:0.61,0) rectangle (axis cs:0.79,5.08123188405797);
\draw[draw=none,fill=color0] (axis cs:1.61,0) rectangle (axis cs:1.79,4.14492770387507);
\draw[draw=none,fill=color0] (axis cs:2.61,0) rectangle (axis cs:2.79,3.98703296703297);
\draw[draw=none,fill=color0] (axis cs:3.61,0) rectangle (axis cs:3.79,4.01028878098646);
\draw[draw=none,fill=color1] (axis cs:-0.19,0) rectangle (axis cs:-0.00999999999999998,3.895);
\draw[draw=none,fill=color1] (axis cs:0.81,0) rectangle (axis cs:0.99,3.895);
\draw[draw=none,fill=color1] (axis cs:1.81,0) rectangle (axis cs:1.99,0.9595);
\draw[draw=none,fill=color1] (axis cs:2.81,0) rectangle (axis cs:2.99,0.973);
\draw[draw=none,fill=color1] (axis cs:3.81,0) rectangle (axis cs:3.99,1.1465);
\draw[draw=none,fill=color2] (axis cs:0.01,0) rectangle (axis cs:0.19,6.48088526864221);
\draw[draw=none,fill=color2] (axis cs:1.01,0) rectangle (axis cs:1.19,6.55193445431113);
\draw[draw=none,fill=color2] (axis cs:2.01,0) rectangle (axis cs:2.19,9.20799043009431);
\draw[draw=none,fill=color2] (axis cs:3.01,0) rectangle (axis cs:3.19,8.33845446012494);
\draw[draw=none,fill=color2] (axis cs:4.01,0) rectangle (axis cs:4.19,13.6530054745664);
\draw[draw=none,fill=color3] (axis cs:0.21,0) rectangle (axis cs:0.39,0.048818146213937);
\draw[draw=none,fill=color3] (axis cs:1.21,0) rectangle (axis cs:1.39,0.0450246145056094);
\draw[draw=none,fill=color3] (axis cs:2.21,0) rectangle (axis cs:2.39,3.42059715333494);
\draw[draw=none,fill=color3] (axis cs:3.21,0) rectangle (axis cs:3.39,3.68364896445113);
\draw[draw=none,fill=color3] (axis cs:4.21,0) rectangle (axis cs:4.39,1.80745877372917);
\draw (axis cs:-0.3,5.1) ++(0pt,3pt) node[
  scale=1.0,
  anchor=south,
  text=black,
  rotate=0.0
]{5.1};
\draw (axis cs:0.7,5.1) ++(0pt,3pt) node[
  scale=1.0,
  anchor=south,
  text=black,
  rotate=0.0
]{5.1};
\draw (axis cs:1.7,4.1) ++(0pt,3pt) node[
  scale=1.0,
  anchor=south,
  text=black,
  rotate=0.0
]{4.1};
\draw (axis cs:2.7,4) ++(0pt,3pt) node[
  scale=1.0,
  anchor=south,
  text=black,
  rotate=0.0
]{4.0};
\draw (axis cs:3.7,4) ++(0pt,3pt) node[
  scale=1.0,
  anchor=south,
  text=black,
  rotate=0.0
]{4.0};
\draw (axis cs:-0.1,3.9) ++(0pt,3pt) node[
  scale=1.0,
  anchor=south,
  text=black,
  rotate=0.0
]{3.9};
\draw (axis cs:0.9,3.9) ++(0pt,3pt) node[
  scale=1.0,
  anchor=south,
  text=black,
  rotate=0.0
]{3.9};
\draw (axis cs:1.9,1) ++(0pt,3pt) node[
  scale=1.0,
  anchor=south,
  text=black,
  rotate=0.0
]{1.0};
\draw (axis cs:2.9,1) ++(0pt,3pt) node[
  scale=1.0,
  anchor=south,
  text=black,
  rotate=0.0
]{1.0};
\draw (axis cs:3.9,1.1) ++(0pt,3pt) node[
  scale=1.0,
  anchor=south,
  text=black,
  rotate=0.0
]{1.1};
\draw (axis cs:0.1,6.5) ++(0pt,3pt) node[
  scale=1.0,
  anchor=south,
  text=black,
  rotate=0.0
]{6.5};
\draw (axis cs:1.1,6.6) ++(0pt,3pt) node[
  scale=1.0,
  anchor=south,
  text=black,
  rotate=0.0
]{6.6};
\draw (axis cs:2.1,9.2) ++(0pt,3pt) node[
  scale=1.0,
  anchor=south,
  text=black,
  rotate=0.0
]{9.2};
\draw (axis cs:3.1,8.3) ++(0pt,3pt) node[
  scale=1.0,
  anchor=south,
  text=black,
  rotate=0.0
]{8.3};
\draw (axis cs:4.1,13.7) ++(0pt,3pt) node[
  scale=1.0,
  anchor=south,
  text=black,
  rotate=0.0
]{13.7};
\draw (axis cs:0.3,0) ++(0pt,3pt) node[
  scale=1.0,
  anchor=south,
  text=black,
  rotate=0.0
]{0.0};
\draw (axis cs:1.3,0) ++(0pt,3pt) node[
  scale=1.0,
  anchor=south,
  text=black,
  rotate=0.0
]{0.0};
\draw (axis cs:2.3,3.4) ++(0pt,3pt) node[
  scale=1.0,
  anchor=south,
  text=black,
  rotate=0.0
]{3.4};
\draw (axis cs:3.3,3.7) ++(0pt,3pt) node[
  scale=1.0,
  anchor=south,
  text=black,
  rotate=0.0
]{3.7};
\draw (axis cs:4.3,1.8) ++(0pt,3pt) node[
  scale=1.0,
  anchor=south,
  text=black,
  rotate=0.0
]{1.8};
\end{axis}
{%
\setlength{\fboxrule}{2.5pt}
\node[align=left] at (9.15,-1.5) {
      \fcolorbox{color0}{white}{Resistenza $[\SI{}{\metre\squared\kelvin\per\watt}]$} \qquad
      \fcolorbox{color1}{white}{Massa superficiale $\times 10^2 \, [\SI{}{\kilo\gram\per\metre\squared}]$} \qquad
      \fcolorbox{color2}{white}{Sfasamento $[\SI{}{\hour}]$} \qquad
      \fcolorbox{color3}{white}{Fattore di attenuazione $\times 10^{-1} [-]$}
      };
}     

\end{tikzpicture}

        \clearpage
        \input{CHAPTERS/confronto.tex}
        \clearpage
        \input{CHAPTERS/confronto2.tex}
\end{landscape}
\end{document}