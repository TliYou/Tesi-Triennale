%!TEX root = ../TesiTriennaleMeoliNicola.tex
\chapter{Introduzione}
Sappiamo come nell'ambito della progettazione edilizia esistano un'infinità di soluzioni costruttive,ma quel è la migliore tra tutte? 
In un precedente lavoro di gruppo era stata analizzata la convenienza economica di diversi pacchetti costruttivi di parete, prendendo come parametri di confronto il costo iniziale, di manutenzione e il tempo di messa in opera. 
Per avere soluzioni simili era stato scelto come parametro fisso la resistenza termica ponendola pari a $R=\SI{4}{m^2K\per W}$.
Il confronto era tra pacchetti costituiti da materiali ben differenti tra loro; tra cui telaio in calcestruzzo armato, telaio in acciaio, muratura portante e parete in X-LAM portante.
A questi strati strutturali si erano poi aggiunti gli isolanti e le rifiniture interne ed esterne.

Per quanto riguarda lo strato strutturale si era arrivati alla conclusione che, dal punto di vista economico, conveniva di più la soluzione in calcelstruzzo armato. 
Perché sebbene richieda molto più tempo di posa in opera, essa è anche molto meno costosa. 
Al lato opposto si aveva la soluzione in X-LAM. 
Ovvero l'unica proposta scelta che prevedeva l'intera parete con una tipologia costruttiva prefabbricata.
Proprio per questo risultava la più rapida, e ciò permetteva di avere una durata del cantiere molto più breve rispetto le altre. 
In più richiedeva meno manutenzione.
Questo era un vantaggio economico nel lungo periodo, ma uno svantaggio enorme iniziale perché il costo di costruzione a causa solo del materiale, era molto più elevato.

(GRAFICO FINALE COSTI O DATI CON I COSTI)

Ora invece si vogliono prendere in esame soltanto due pacchetti e confrontarli non più dal punto di vista economico, ma da quello termo-igrometrico.
Si vuole cercare di capire, cioè, se nonostante i costi più elevati iniziali, ci siano dei vantaggi che facciano scegliere per il pacchetto composto da legno.

Nei capitoli successivi si entrerà più nel dettaglio, riportando un'analisi di Glaser effettuata e una ricerca accademica su questo materiale diametralmente opposto a tutti gli altri.

(DA AGGIUNGERE QUALCHE DATO DELLA VECCHIA RELAZIONE)