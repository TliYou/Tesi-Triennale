%!TEX root = ../TesiTriennaleMeoliNicola.tex
\chapter{Analisi dinamica}
\section*{Alcune definizioni e nomenclature}
Con la denominazione dell'á'pice $^{(k)}$ d'ora in avanti si farà riferimento ad una quantità che và calcolata per ogni strato $k$-esimo. 
Con il  grassetto verranno indicate delle matrici, mentre con un doppio pedice le componenti di tali matrici.
\begin{table}[H]
\centering
\begin{tabular}{p{0.3\textwidth}p{0.7\textwidth}}
    \toprule
    \textbf{Profondità di penetrazione}  \[\delta^{(k)} = \sqrt{\frac{\lambda^{(k)} \, T}{\pi \,\rho^{(k)} \,c^{(k)}}} \,\left[\SI{}{\metre}\right]\] & definizione definizione definizione definizione definizione definizione definizione definizione definizione definizione definizione definizione definizione definizione definizione \\
    \textbf{Rapporto lunghezze} \[\xi^{(k)} = \dfrac{s^{(k)}}{\delta^{(k)}} \,\left[\SI{}{-}\right] \] & definizione definizione definizione definizione definizione definizione definizione definizione definizione definizione definizione definizione definizione definizione definizione \\
    \bottomrule
\end{tabular}
\end{table}
\section*{Metodologia secondo la normativa}
La matrice di trasferimento, calcolata per ogni singolo strato $k$-esimo è definita come:
\begin{equation}
    \mathbf{Z}^{(k)} =
    \begin{bmatrix}  
        Z_{11}^{(k)} & Z_{12}^{(k)} \\
        Z_{21}^{(k)} & Z_{22}^{(k)}
    \end{bmatrix}
\end{equation}
dove le componenti della matrice $\mathbf{Z^{(k)}}$ sono

{\footnotesize
\begin{align*}
    Z_{11}^{(k)} &= Z_{22}^{(k)} = \cosh\left(\xi^{(k)}\right)\cos\left(\xi^{(k)}\right) + \mathrm{i} \left[\sinh\left(\xi^{(k)}\right)\sin\left(\xi^{(k)}\right)\right] \\
    Z_{12}^{(k)} &=-\frac{\delta^{(k)}}{2\lambda^{(k)}}\Bigl\{\sinh\left(\xi^{(k)}\right)\cos\left(\xi^{(k)}\right)+\cosh\left(\xi^{(k)}\right)\sin\left(\xi^{(k)}\right) + \mathrm{i}\left[\cosh\left(\xi^{(k)}\right)\sin\left(\xi^{(k)}\right)-\sinh\left(\xi^{(k)}\right)\cos\left(\xi^{(k)}\right)\right]  \Bigr\} \\
    Z_{21}^{(k)} &= \frac{\lambda^{(k)}}{\delta^{(k)}}\Bigl\{  \sinh\left(\xi^{(k)}\right)\cos\left(\xi^{(k)}\right)-\cosh\left(\xi^{(k)}\right)\sin\left(\xi^{(k)}\right) + \mathrm{i} \left[\sinh\left(\xi^{(k)}\right)\cos\left(\xi^{(k)}\right)+\cosh\left(\xi^{(k)}\right)\sin\left(\xi^{(k)}\right)\right] \Bigr\}
\end{align*}
}

Moltiplicando le matrici di trasferimento di ogni singolo strato si ottiene la matrice di trasferimento dell'intero componente edilizio relativa agli $N$  strati:
\begin{equation}
    \mathbf{Z}=\mathbf{Z}_{N}\cdot\mathbf{Z}_{N-1}\cdot\ldots\cdot\mathbf{Z}_{2}\cdot\mathbf{Z}_{1} \qquad .
\end{equation}
Calcolando le matrici di trasferimento degli strati liminari come
\begin{equation}
    \mathbf{Z}_{\text{int.}} = 
    \begin{bmatrix}
        1 & -R_{\text{s,int.}} \\
        0 & 1
    \end{bmatrix}
    \qquad
    \mathbf{Z}_{\text{est.}} = 
    \begin{bmatrix}
        1 & -R_{\text{s,est.}} \\
        0 & 1
    \end{bmatrix}
\end{equation}
si può infine ottenere la matrice di trasferimento da ambiente ad ambiente:
\begin{equation}
    \mathbf{Z}_{\text{ee}} = \mathbf{Z}_{\text{est.}} \cdot \mathbf{Z} \cdot \mathbf{Z}_{\text{int.}} \qquad .
\end{equation}

Dagli elementi di questa matrice è possibile ricavare le grandezze di nostro interesse per l'analisi dinamica.
Dall'elemento {\footnotesize $12$} della matrice $\mathbf{Z_{\text{ee}}}$ si ricava la trasmittanza termica periodica $Y_{12}$ tramite il modulo della parte reale e immaginaria
\begin{equation}
    Y_{12} = \left\lvert -\frac{1}{Z_{\text{ee},12}}\right\rvert 
\end{equation} 
e il fattore di attenuazione $f$
\begin{equation}
    f = -\frac{Y_{12}}{U} \qquad .
\end{equation}
Lo sfasamento riferito al tempo iniziale si ottiene infine come:
\begin{equation}
    \Delta T = \frac{T}{2\,\pi}\arg\left(Z_{\text{ee},12}\right) + \frac{T}{2} \qquad .
\end{equation}
Dagli elementi {\footnotesize $11$} e {\footnotesize $22$} si ricavano riispettivamente l'ammettenza termica interna ed esterna 
\begin{equation}
    Y_{\text{ii}}=Y_{11} = \left\lvert -\frac{1}{Z_{\text{ee},11}}\right\rvert \qquad \qquad
    Y_{\text{ee}}=Y_{22} = \left\lvert -\frac{1}{Z_{\text{ee},22}}\right\rvert \qquad .
\end{equation} 
Un'altra quantità molto utile è la capacità termica periodica interna per unità di superficie. \e definita dalla normativa come $k_1 = \frac{T}{2\,\pi} \left\lvert Y_{11} - Y_{12}\right\rvert$ e sostituendo i termini in funzione dalla matrice di trasfermento, si può riscrivere come:
\begin{equation}
    k_1 = \frac{T}{2\,\pi} \left\lvert \frac{Z_{\text{ee},11} - 1}{Z_{\text{ee},12}}\right\rvert\,.
\end{equation}
